
% IUT-Thesis v2.5
%CHANGE LOG IUT-Thesis v2.5
%---------------------------
%- Add some options to signature pages
%- Adaption to Texlive 2016, Xeperian 18.2 and Bidi 30.2
%- Fix some buges: paragraph footnote, reference style, and ...


% این قالب بر اساس فرمت پایان‌نامه‌ها و رساله‌های تحصیلات تکمیلی دانشگاه صنعتی اصفهان تهیه شده است. شما می‌توانید آخرین نسخه این قالب را از دریافت کنید.

% توصیه می‌شود که از توزیع تک‌لایو (TexLive) استفاده شود:
% http://tug.org/texlive/acquire-iso.html

% موفق باشید.
% محمد جان‌نثاری
% mohammad.jannesari@gmail.com
% امین فخاری
% a101.fakhari@gmail.com
% ‌فروردین 1396
% -----------------------------------------------------------------------------------

% نکات:

% برای دریافت نتیجه مطلوب از این قالب بایستی از آخرین نسخه تک‌لایو به همراه Xepersian نسخه 18.2 , Bidi نسخه 30.2 استفاده شود.

% برای آن‌که پردازش فایل و مشاهده خروجی در هنگام نوشتن پایان‌نامه آسان‌تر و سریع‌تر انجام شود، انجام موارد زیر توصیه می گردد:
% الف) فصل‌ها و بخش‌هایی که در حال نوشتن آن‌ها نیستید را غیر فعال کنید. به‌عنوان مثال، در این قالب، این دستورات را می‌توان در صورت عدم نیاز با اضافه کردن % به طور موقت غیرفعال کرد:
% \MakeTitlePage
% \MakeFarsiSignaturePage
% \input{Chapters/Acknowledgments}
% \MakeCopyRightPage
% \input{Chapters/Dedication}
% \MakeTableOfContents
% \MakeListOfFigures
% \MakeListOfTables
% \MakeFarsiAbstract
% \input{Chapters/Chapter#}
% \MakeAppendices
% \input{Chapters/Appendices}
% \MakeEnglishAbstract
% \MakeEnglishSignaturePage
% ب) از گزینه draft برای فراخوانی کلاس استفاده کنید. یعنی
% \documentclass[a4paper,fleqn,10pt,oneside,draft]{book}
% این گزینه حالت چرکنویس را ایفا می‌کند و بر روی بسته‌های مختلف اثرهای متفاوتی دارد. به‌عنوان مثال: به جای شکل، تنها چهارچوب آن نمایش داده شود، لینک‌های hyperref غیر فعال گردد، فایل‌های خارجی را در بسته listings اضافه نمی‌کند و ... و همه این موارد سبب کاهش زمان اجرا و حجم فایل می‌شود.

% در صورتی که میخواهید به سطر بعد بروید اما نمیخواهید بین دو کلمه‌ای که نوشتید فاصله بیفتد کافی است در انتهای خط اول  (بدون فاصله) کاراکتر % را اضافه کنید. با این عمل، لاتک خط فاصله ایجاد شده در اثر تغییر سطر را به عنوان توضیح اضافه یا کامنت در نظر میگیرد و در خروجی اعمال نمی‌کند.

% توصیه می‌شود از شکل‌های برداری با فرمت PDF استفاده شود. این کار علاوه بر افزایش کیفیت رسال/پایان‌نامه/گزارش، باعث کاهش حجم شکل‌ها (و در نتیجه  کاهش حجم فایل نهایی) و همچنین کاهش زمان پردازش می‌شود.

% در این قالب سعی شده است که از تمامی بخش‌های موجود در پایان‌نامه‌ها نمونه‌ای آورده شود.

% لطفا هرگونه عدم تطابق این قالب با فرمت دانشگاه صنعتی اصفهان را به ایمیل (mohammad.jannesari@gmail.com) اطلاع دهید.

\documentclass[a4paper,fleqn,10pt,oneside]{book}
\usepackage{Settings/IUT-Thesis}
%-----------------------------
% دستورهای مورد نیاز را در این قسمت اضافه نمایید:
\allowdisplaybreaks
%-----------------------------

\begin{document}

\pagestyle{plain}
\pagenumbering{adadi}
\setcounter{page}{2}

% ░░░░░░░▒▒▒▒▒▒▓▓▓▓ In the Name of Allah ▓▓▓▓▒▒▒▒▒▒░░░░░░░
\clearpage
\thispagestyle{empty}
\begin{figure}[t]
\centering
\includegraphics[scale=1.3]{Settings/Allah.pdf}
\end{figure}

% ░░░░░░░▒▒▒▒▒▒▓▓▓▓ Title Page ▓▓▓▓▒▒▒▒▒▒░░░░░░░
\DepartmentFa{مهندسی برق و کامپیوتر  }
\ThesisTypeFa{پایان‌نامه} % Or \ThesisTypeFa{رساله} Or \ThesisTypeFa{پیشنهادیه پایان‌نامه}
\DegreeFa{کارشناسی} % Or \DegreeFa{دکتری} 
\FieldFa{مهندسی کامپیوتر}
\YourFullnameFa{نامی نذیری}
\FirstSupervisorFa{دکتر مازیار پالهنگ}
\YearFa{1401}
\TitleFa{
بررسی سیستم گراف انیمیشن در موتور بازی سازی آنریل و 
\\[0.4cm]
و پیاده‌سازی یک سیستم انیمیشن با استفاده از 
\lr{ OpenGL}
}
% اگر عنوان رساله طولانی بود، در دو خط به صورت نشان داده شده تقسیم شود.

\MakeTitlePage%

% ░░░░░░░▒▒▒▒▒▒▓▓▓▓ Signature - Farsi ▓▓▓▓▒▒▒▒▒▒░░░░░░░
\Prefix{آقای} %\Prefix{خانم}
\DateFa{1395/10/20}
\FirstExaminerFa{دکتر داور اول} % Optional (Remove It If You Don't Have)
\SecondExaminerFa{دکتر داور دوم} % Optional (Remove It If You Don't Have)
\DeanOfDepartmentFa{دکتر تحصیلات تکمیلی دانشکده}

\MakeFarsiSignaturePage%

% ░░░░░░░▒▒▒▒▒▒▓▓▓▓ Acknowledgments ▓▓▓▓▒▒▒▒▒▒░░░░░░░
%todo\input{Chapters/Acknowledgments}%

% ░░░░░░░▒▒▒▒▒▒▓▓▓▓ CopyRight ▓▓▓▓▒▒▒▒▒▒░░░░░░░
\MakeCopyRightPage%

% ░░░░░░░▒▒▒▒▒▒▓▓▓▓ Dedication ▓▓▓▓▒▒▒▒▒▒░░░░░░░
%todo\input{Chapters/Dedication}%

% ░░░░░░░▒▒▒▒▒▒▓▓▓▓ Table of Contents/Figures/Tables ▓▓▓▓▒▒▒▒▒▒░░░░░░░
\MakeTableOfContents%
\MakeListOfFigures%
\MakeListOfTables%
\MakeListOfAlgorithms% for test

% ----------------------------------------------------------------------------
\clearpage
\pagestyle{myheadings}
\pagenumbering{arabic}
\setcounter{page}{1}

% ░░░░░░░▒▒▒▒▒▒▓▓▓▓ Abstract - Farsi ▓▓▓▓▒▒▒▒▒▒░░░░░░░
%todo\AbstractFa{
    پویانمایی کامپیوتری فرایندی است که برای تولید تصاویر متحرک دیجیتالی استفاده می‌شود. پویانمایی کامپیوتری مدرن معمولا از گرافیک کامپیوتری سه‌بعدی برای ایجاد یک تصویر سه‌بعدی استفاده می‌کند. در اکثر سیستم‌های پویانمایی کامپیوتری سه‌بعدی یک انیماتور نمایش ساده از آناتومی یک شخصیت ایجاد می‌کند که مشابه یک اسکلت یا آدمک می‌باشد. در شخصیت‌های انسان و حیوانات اکثر قسمت‌های این مدل اسکلتی با استخوان‌های واقعی مطابقت دارد.
    \\
    گام اول این پروژه بررسی سیستم گراف پویانمایی در موتور بازی‌سازی آنریل می‌باشد. گراف پویانمایی برای محاسبه‌ی وضعیت نهایی یک مش اسکلتی در فریم فعلی استفاده می‌شود. به صورت کلی این گراف برای نمونه‌گیری ، ترکیب و دستکاری ژست‌ها استفاده می‌شود و این ژست به مش‌های اسکلتی توسط طرح پویانمایی اعمال می‌شود. در این گام به بررسی این گراف و الگوریتم‌های به کار ‌گرفته ‌شده در آن خواهیم‌ پرداخت. 
    \\
    در گام دوم نیز به پیاده‌سازی سیستم پویانمایی اسکلتی از پایه، با توجه به روش‌های بدست‌آمده پرداخته می‌شود. این مرحله سه هدف را دنبال می‌کند. هدف اول نمایش اسکلتون در یک محیط سه‌بعدی و نورپردازی آن، که با استفاده از 
    \lr{OpenGL}
    به ‌وجود آمده است، می‌باشد. در این مرحله باید با استفاده از زبان 
    \lr{C++} 
    برنامه‌ای بنویسیم که در نهایت بتواند یک کلیپ پویانمایی به‌ وجود آمده به وسیله‌ی فریم‌های کلیدی را نمایش دهد. هدف دوم اضافه کردن یک مش به اسکلتون با استفاده از روش های پوسته سازی می‌باشد. هدف نهایی نیز پیاده‌سازی روش ترکیب کلیپ‌های پویانمایی می‌باشد تا بتوانیم کلیپ‌های مختلف را با یکدیگر ترکیب کنیم. 
}

\KeywordsFa{
 پویانمایی کامپیوتری،  موتور بازی‌سازی ، موتور آنریل،  گراف پویانمایی، گرافیک سه‌بعدی کامپیوتری
}%
\MakeFarsiAbstract%

% ░░░░░░░▒▒▒▒▒▒▓▓▓▓ Chapters ▓▓▓▓▒▒▒▒▒▒░░░░░░░
\clearpage
\baselineskip=0.9cm

\chapter{مقدمه}
پویانمایی هنر جان‌بخشیدن به اجسام بدون جان است.

والت دیزنی درباره‌‌ی پویانمایی می‌گوید: "انیمیشن می تواند هر آنچه را که ذهن انسان تصور می‌کند، توضیح دهد"
وقتی می‌گوییم جسمی را پویا کردیم، یعنی به آن جان بخشیدیم.
زمانی که یک فیلم پویانمایی شده را در تلوزیون یا سینما می‌بینید، شخصیت‌های درون آن فیلم در حالت حرکت هستند.
این حرکت معمولا صاف و به هم پیوسته است. نوارهای حاوی فیلم متشکل از دنباله‌ای از تصاویر هستند که به عنوان "فریم" شناخته می‌شوند و درواقع با پخش شدن این فریم‌ها
به صورت متوالی، توهم ایجاد حرکت به مخاطب القا می‌شود.

پویانمایی تاریخچه‌ای غنی‌ دارد. در این فصل ابتدا به بررسی این تاریخچه با توضیحاتی 
درباره‌ی پویانمایی سنتی و پس از آن پویانمایی کامپیوتری پرداخته می‌شود.
پس از آن به بررسی روش‌های کلی که توسط هنرمندان برای ایجاد پویانمایی به‌کارگرفته می‌شوند پرداخته می‌شود.
در نهایت به بررسی به کارگرفتن این انیمیشن‌ها در موتورهای بازی سازی پرداخته می‌شود.

\section{تاریخچه‌ی پویانمایی سنتی}

پویانمایی سنتی که با اسم‌های مختلفی مانند "پویانمایی مرسوم" ، "پویانمایی سل‌ای"و "پویانمایی بادست" شناخته می‌شود، روشی 
غالب برای تولید فیلم‌های پویانمایی‌شده در حدود قرن 20 میلادی بود.
در این روش، به صورت کلی پویانمایی به وسیله‌ی نقاشی با دست به وجود می‌‌آمد.
درواقع هر فریم از فیلم، یک عکسی از نقاشی بود.
برای به وجود آوردن توهم حرکت، هر نقاشی اندکی با نقاشی قبلی خود تفاوت داشت.

برای تولید پویانمایی سنتی، از روش‌های مختلفی استفاده می‌شد. در اینجا به بررسی
سه عدد از این روش‌ها می‌پردازیم.

\subsection{فریم‌های کلیدی و درمیان}
از آنجایی که تولید پویایی با دست و کشیدن نقاشی کار بسیار طولانی‌ای بود، برای اینکه وقت پویانمای‌های ارشد 
ذخیره شود، این پویانماها فریم‌های اصلی یک حرکت را بر روی کاغذ ترسیم می‌کردند و 
فریم‌های میانی را پویانماهای جوان پر می‌کردند.

\begin{figure}[ht]
	\centerline{\includegraphics[width=\textwidth,height=\textheight,keepaspectratio]{Figures/Ch1/KeyframeAnimation.png}}

	\caption{فریم‌های کلیدی و درمیان}
	\label{fig:KeyframeAnimation}
\end{figure}

\subsection{چشم‌انداز چندمنظوره}

استفاده از چشم‌انداز چندمنظوره روش دیگری بود که در پویانمایی سنتی استفاده می‌شد.
هماطور که از تصویر زیر مشخص است، برای نمایش یک محیط از یک چشم‌انداز استفاده می‌شد.
این چشم‌انداز می‌توانست نشان دهنده‌ی محیط در فواصل مختلف باشد. در این صورت، زمانی که 
دوربین در صحنه حرکت می‌کرد این توهم را در مخاطب ایجاد می‌کرد که گویی در محیط در حال حرکت هستیم.

\begin{figure}[ht]
	\centerline{\includegraphics[width=\textwidth,height=\textheight,keepaspectratio]{Figures/Ch1/Panorama.png}}

	\caption{چشم‌انداز چندمنظوره}
	\label{fig:Panorama}
\end{figure}


\subsection{لایه‌های مختلف}

با استفاده از این روش، پویانما‌ها یک صحنه را به چند قسمت مختلف تقسیم می‌کردند.
به صورت مثال لایه‌های مختلف برای هر شخصیت درون صحنه استفاده می‌شد. علاوه برا ین یک لایه نیز برای تصویر پس‌زمینه استفاده می‌شد.
از آنجایی که این لایه‌ها یک صفحه‌ی شفاف بودند بنابراین می‌توان لایه‌‌ها را 
بر روی هم انباشته کرد و با تصویر برداری از بالا تمام صحنه را تصویربرداری کرد.
این روش در تصویر 
\ref{fig:DifferentLayers}
آورده شده است.

\begin{figure}[ht]
	\centerline{\includegraphics[width=\textwidth,height=\textheight,keepaspectratio]{Figures/Ch1/DifferentLayers.png}}

	\caption{لایه‌های مختلف}
	\label{fig:DifferentLayers}
\end{figure}

\section{پویانمایی کامپیوتری}

اگر بخواهیم نگاهی به تاریخچه‌ی انیمیشن‌های کامپیوتری بیاندازیم، مشاهده می‌کنیم که 
در حدود دهه‌ی 1980 میلادی شرکت دیزنی به عنوان یکی از اولین شرکت‌های جهان، شروع به 
دیجیتالی کردن خط لوله‌ی تولید پویانمایی سنتی خود کرد.
در این دیجیتال‌سازی بسیاری از روش‌ها و ایده‌‌های استفاده شده در پویانمایی سنتی،
به‌کار گرفته‌شد.
اولین مقالات این حوزه توسط آقای جان لستر از کارمندان پیکسار به عنوان 
"اصول پویانمایی سنتی به‌کار رفته در پویانمایی کامپیوتری سه‌بعدی"
ارائه شد.
در این مقاله اصول اولیه پویانمایی سنتی دوبعدی ترسیم شده با دست
و کاربرد آن‌ها در پویانمایی کامپیوتری سه‌بعدی شرح داده شده است.

پویانمایی کامپیوتری تنها محدود به دنیای سینما و فیلم‌های پویانمایی نمی‌شوند و به دنیای
بازی‌های کامپیوتری نیز ورود پیدا کرده‌اند. بازی‌های کامپیوتری سعی می‌کنند دیوار میان تماشاگران و فیلم را بشکنند و 
با تعاملی بودن و دادن آزادی عمل به بازیکن، سعی می‌کنند داستان را به گونه‌ای تعریف کنند که گویی بازیکن یکی از شخصیت‌های اصلی داستان است.
پویانمایی در بازی‌های کامپیوتری اهمیت بسیار بالایی دارد زیرا همانطور که گفته شد باعث 
جان بخشیدن به شخصیت‌ها می‌شود که اهمیت بسیار بالایی برای جلب توجه بازیکنان در هنگام داستان‌سرایی دارد.

با پیشرفت تکنولوژی، همراه با استفاده از روش‌های گذشته، روش‌های جدیدتری برای تولید پویانمایی توسعه یافته‌است که 
در ادامه به چند مورد از آن‌‌ها می‌پردازیم.

\subsection{فریم‌های کلیدی و درمیان}

همانطور که اشاره شد در پویانمایی کامپیوتری از روش‌های موجود در 
پویانمایی سنتی استفاده شده است. در اینجا نیز فریم‌های کلیدی 
یک حرکت توسط پویانماها به وجود می‌‌آیند ولی فریم‌های میانی به جای اینکه توسط پویانماها به وجود آید،
توسط کامپیوتر با استفاده از روش های درون‌یابی به وجود می‌آیند.

\subsection{رویه}

در این روش، حرکت بر اساس یک الگوریتم بیان می‌شود.
درواقع انیمیشن‌ها در این نوع پویانمایی، توابعی با تعداد کمی از متغیر‌ها هستند.
به عنوان مثال یک تابعی را درنظر بگیرید که به گرفتن ورودی ثانیه، دقیقه و ساعت، 
یک شئ ساعت را خروجی دهد که عقربه‌هایش در جای مناسب با توجه به ورودی‌ها قرار گرفته باشد.
حال می‌توان با تغییر ورودی‌ها حرکت ساعت را شبیه‌سازی کنیم.

\subsection{مبتنی بر فیزیک}

پویانمایی مبتنی بر فیزیک پلی میان دنیای پویانمایی با 
دنیای واقعی است. در این روش با نسبت دادن ویژگی‌های فیزیک به اشیاء سه‌بعدی و سپس حل‌کردن
فرمول‌های فیزیک مانند فرمول حرکت یا فرمول‌های نیوتن،
فیزیک را شبیه سازی می‌کند.
پویانمایی‌های مبتنی بر فیزیک شخصیت را قادر می‌سازد تا حرکت‌های خود را 
به صورت پویا با محیط تنظیم کند.

\subsection{ضبط حرکت
\protect \LTRfootnote{Motion Capture}}

به فرآیند ثبت و دیجیتالی‌کردن حرکت یک شئ یا شخص، ضبط حرکت گویند.
ضبط حرکت توسط دوربین‌های مادون قرمز که تعدادی زیادی از آن‌ها در صحنه‌ی ضبط قرار دارند، صورت می‌گیرد.
این دوربین‌ها به صورت شبکه به یکدیگر متصل هستند و پس از کالیبره شدن، آماده‌ی استفاده هستند.
این دوربین‌ها با استفاده از نشانگر‌های سفیدی که بر روی لباس بازیگران 
ضبط حرکت قرار دارد، داده‌های مورد نیازشان را دریافت می‌کنند.
قابل ذکر است این نشانگر‌ها بازتابنده‌ی مادون قرمز هستند.
در نهایت پویانماها به پاکسازی و پردازش این داده‌ها پرداخته تا آن را 
برای استفاده‌ی شخصیت‌های سه بعدی آماده کنند.



\section{موتور بازی‌سازی}
موتور‌های بازی‌سازی پلتفرم‌هایی هستند که ساخت بازی‌های رایانه‌ای را آسان‌تر می‌کنند.
آن ها به شما این امکان را می‌دهند تا عناصر بازی مانند انیمیشن، تعامل با کاربر یا تشخیص برخورد میان اشیاء را در یک واحد ادغام و ترکیب کنید.
\cite{barczak2019comparative}
زمانی که از اصطلاح موتور بازی‌سازی استفاده می‌کنیم منظورمان نرم‌افزارهای قابل توسعه‌ای هستند که می توانند پایه و اساس بسیاری از بازی‌های مختلف باشند.
\cite{GameEngineArchitecture}
موتورهای بازی‌سازی متشکل از اجزای مختلفی هستند که قابلیت‌های لازم برای ساخت بازی را فراهم می‌کنند.
رایج ترین اجزای موتور بازی عبارتند از:
\cite{barczak2019comparative}
\begin{itemize}
    \item[-] مولفه‌ی صدا: نقش اصلی این مولفه تولید جلوه‌های صوتی در بازی است.
    \item[-] موتور رندر: وظیفه اصلی این مولفه تبدیل داده‌های ورودی به پیکسل‌ها، برای به تصویر کشاندن بر روی صفحه است.
    \item[-] مولفه هوش مصنوعی: این مولفه مسئولیت ارائه‌ی تکنیک‌هایی برای تعریف قوانین رفتار شخصیت‌هایی را دارد که توسط بازیکنان کنترل نمی‌شوند.
    \item[-] مولفه انیمیشن: نقش اصلی این مولفه اجرای انیمیشن‌های مختلف مانند حرکت است.
    \item[-] مولفه شبکه: وظیفه اصلی این مولفه قادرساختنِ بازیِ همزمانِ بازیکنان با یکدیگر، از طریق استفاده از دستگاه‌های متصل به اینترنت است.
    \item[-] مولفه منطق یا مکانیک بازی: این مولفه قوانین حاکم بر دنیای مجازی، ویژگی‌های شخصیت‌های بازیکنان، هوش مصنوعی و اشیاء موجود در دنیای مجازی و همچنین وظایف و اهداف بازیکنان را تعریف می‌کند.
    \item[-] ابزارهای نرم‌افزاری: وظیفه اصلی این ابزارها افزایش راندمان و سرعت تولید بازی با موتور بازی‌سازی است. آن‌ها توانایی اضافه‌کردن بسیاری از عناصر مختلف را به بازی‌ها، از انیمیشن و جلوه‌های صوتی گرفته تا الگوریتم‌های هوش مصنوعی، را فراهم می‌کنند.   
\end{itemize}

یکی از مهم‌ترین مولفه‌های موجود در هر موتور بازی، مولفه‌ی انیمیشن آن است. در این پروژه به بررسی سیستم
انیمیشن گراف که وظیفه‌ی پخش انیمیشن‌ شخصیت‌های سه‌بعدی را در موتور بازی آنریل دارد می‌پردازیم.



\section {موتور بازی‌سازی آنریل}

اولین نسل موتور بازی‌سازی آنریل توسط تیم سوینی، بنیانگذار اپیک گیمز
\LTRfootnote {Epic Games}
،
توسعه یافت.
سویینی در سال 1995 شروع به نوشتن این موتور برای تولید بازی‌ تیراندازی اول شخصی به اسم غیرواقعی
\LTRfootnote{Unreal}
کرد.


نسخه‌‌ی دوم موتور بازی‌سازی آنریل در سال 2002 منتشر شد. 

نسخه سوم نیز در سال 2004 پس از حدود 18 ماه توسعه، منتشر شد.
در این نسخه، معماری پایه‌ای موجود در نسخه‌ی اول مانند طراحی شی‌گرا، اسکریپت‌نویسی مبتنی بر داده و رویکرد نسبتا ماژولار نسبت به زیرسیستم‌ها وجود داشت.
اما برخلاف نسخه دوم که از یک خط لوله با عملکرد ثابت
\LTRfootnote{fixed-function pipeline}
استفاده می‌کرد، این نسخه به صورتی طراحی شده بود تا بتوان قسمت‌های سایه‌زنی سخت‌افزاری
\LTRfootnote{shader hardware}
را برنامه‌نویسی کرد.


موتور بازی‌سازی آنریل 4 در سال 2014 در کنفرانس توسعه‌دهندگان بازی
\LTRfootnote{GDC}
منتشر شد.
این نسخه با طرح کسب‌و‌کار اشتراکی برای توسعه‌دهندگان در دسترس قرار گرفت. این اشتراک به صورت ماهانه، با پرداخت 19 دلار آمریکا به توسعه‌دهندگان این اجازه را می‌داد تا به نسخه‌ی کامل موتور، از جمله کد منبع 
\lr {C++}
آن
دسترسی پیدا‌ کنند.
البته در سال 2015 اپیک گیمز موتور بازی‌سازی آنریل را به صورت رایگان برای همگان منتشر ساخت.

آخرین نسخه آنریل به اسم موتور بازی‌سازی آنریل 5 در سال 2020 معرفی شد. این نسخه از تمام سیستم‌های موجود از جمله کنسول‌های نسل بعدی پلی‌استیشن 5
\LTRfootnote{PlayStation 5}
و ایکس‌باکس سری 
\lr{X/S}
\LTRfootnote{Xbox Series X/S}
پشتیبانی می کند.
کار بر روی این موتور حدود دو سال قبل از معرفی آن شروع شده بود. در سال 2021 نسخه‌ای از آن به صورت دسترسی اولیه منتشر شد. به طور رسمی در سال 2022 نسخه‌ی کامل این موتور برای توسعه‌دهندگان انتشار یافت.
\cite{UnrealEngineWikiPedia}

\chapter { انیمیشن‌های اسکلتونی }

%{
%در این فصل برای پیا‌ده‌سازی سیستم انیمیشن به دنبال ارضا کردن 4 هدف کلی هستیم.
%\begin{enumerate}
%	\item نمایش اسکلت شخصیت در یک محیط سه‌بعدی
%	\item نمایش انیمیشن‌های سه‌بعدی اسکلت
%	\item نمایش شخصیت دارای مدل و اجرای انیمیشن بر روی آن
% 	\item با پیاده‌سازی روش ترکیب برای ترکیب انیمیشن‌های مختلف با یکدیگر
%\end{enumerate}

انیمیشن‌های اسکلتونی تکنیکی در انیمیشن‌های کامپیوتری است که در آن شخصیت درون بازی به دوبخش تقسیم ‌می‌شود. یک بخش،یک مش یا پوسته است که برای به نمایش کشاندن ‌آن شخصیت در محیط سه‌بعدی استفاده می‌شود و بخش دوم یک اسکلت است. این اسکلت مجموعه سلسله مراتبی از قطعات به‌ هم پیوسته است که به هر قطعه یک مفصل گویند.
در این فصل به بررسی این دوبخش و تکنیک‌های موجود در انیمیشن‌های اسکلتونی خواهیم پرداخت

\section{مدل اسکلتونی}

در انیمیشن‌های اسکلتونی از مدل‌های اسکلتونی استفاده می‌شود. هر مدل اسکلتونی از دو بخش مدل و اسکلت تشکیل شده است. 

\section{شبکه‌ی\protect\LTRfootnote{Mesh} چندضلعی}

.در گرافیک کامپیوتری سه‌بعدی و مدل‌سازی جامد، شبکه چند‌ضلعی مجموعه‌ای از رئوس، لبه‌ها و وجوه است که شکل یک جسم چند‌وجهی را مشخص می‌کند
وجه‌ها معمولاً از مثلث‌ها (شبکه مثلثی)، چهار ضلعی (چهار گوشه)، یا دیگر چند ضلعی‌های محدب ساده
(
	\lr{n}
	ضلعی‌ها
)
تشکیل شده‌اند. دلیل استفاده از این نوع چند ضلعی‌ها آسان‌تر بودن به نمایش کشیدن وجوه در محیط سه‌بعدی است.
البته در حالت کلی اشیاء ممکن است از چندضلعی‌های مقعر و یا حتی چندضلعی‌های دارای سوراخ نیز تشکیل شده باشند.

اشیاء ایجادشده توسط مش‌های چند ضلعی باید انواع مختلفی از عناصر، از جمله رئوس، لبه‌ها، وجوه، چندضلعی‌ها و سطوح را ذخیره کنند.
در بسیاری از نرم‌افزارهای سه‌بعدی، فقط رئوس، لبه‌ها و یکی از دو مورد وجوه یا چند‌ضلعی‌ها ذخیره می‌شوند.
در اکثر سیستم‌های رندرکننده
\LTRfootnote{renderer}
فقط از وجوه سه‌ضلعی
(مثلث‌ها)
استفاده‌ می‌شود.
بنابراین در این حالت چند‌ضلعی‌های مدل باید به شکل مثلث باشند. البته سیستم‌های رندرای وجود دارند که از چهارضلعی‌ها یا چندضلعی‌های با تعداد اضلاع بالاتر نیز پشتیبانی ‌می‌کنند و یا در لحظه این چندضلعی‌ها را به مجموعه‌ای از مثلث‌ها تبدیل می‌کنند که در این صورت باعث ‌می‌شود نیازی به ذخیره‌ی مش به شکل مثلثی نباشد.

بنابراین چهار قسمت اصلی یک مش چندضلعی، رئوس، لبه‌ها، وجوه و چندضلعی‌ها هستند. توضیح کوتاهی درباره‌ی هر کدام از این موارد را در بخش زیر می‌توانیم مشاهده کنیم.

\subsection{راس}
راس‌ها معمولا یک موقعیت در فضای سه‌بعدی همراه با اطلاعات دیگر مانند رنگ، بردار نرمال، مختصات بافت 
در راس‌های مربوط به مش‌های اسکلتونی اطلاعاتی مانند تعداد مفاصلی که بر روی این راس تاثیر می‌گذارد همراه با وزن تاثیرگذاری‌اش می‌تواند اضافه شود.

\subsection{لبه}
ارتباط بین دو راس را لبه گویند.

\subsection{وجه}
مجموعه‌ای بسته از لبه‌ها را وجه گویند. وجه‌ها می‌توانند از سه لبه 
(وجه مثلثی)
یا از چهار لبه
(وجه چهارگوش)
تشکیل شده باشند.

\subsection{چندضلعی}
یک چندضلعی مجموعه‌ای همسطح از وجود است.
در سیستم‌هایی که از وجه‌های چند ضلعی پشتیبانی می‌کنند، وجوه و چندضلعی‌ها یکسان هستند ولی در صورتی که سیستم مورد نظر تنها از سه یا چهار ضلعی‌ها پشتیبانی کند، در این صورت چند ضلعی‌ها را مجموعه‌ای از وجوه گویند.

\section{مدل}

مدل‌
\footnote{ گاهی به جای استفاده از واژه‌ی مدل، از واژه‌ی مش هم استفاده می‌شود.}
درواقع هر شئ‌ای است که در محیط سه‌بعدی قرار می‌گیرد و به تصویر کشیده ‌می‌شود. هر مدل می‌تواند از چند زیرمش تشکیل شود.
به عنوان مثال یک ماشین را درنظر بگیریم. موجودیت ماشین می‌تواند یک مدل باشد که در محیط سه‌بعدی قرار می‌گیرد. مدل ماشین می‌تواند از چند زیرمش مانند چرخ‌ها، لاستیک‌ها و بدنه‌ی ماشین تشکیل شود. دلیل وجود داشتن یک موجودیت کلی به اسم ماشین این است که یک ‌شخصی مانند طراح محیط و یا طراح مرحله ‌نمی‌خواهد هر بار که ماشینی را در محیط قرار دهد، تک تک زیرمش‌ها را به صورت دستی در صحنه وارد کند و در سر جای خودش قرار بدهد.


\section{زیرمش
\protect\LTRfootnote{Sub-Mesh}
}
چندضلعی‌های دارای یک نوع ماده
\LTRfootnote{Material}
را یک زیرمش گویند.
همانطور که اشاره شد، هر مدل از چند زیرمش تشکیل می‌شود. دلیل این تقسیم این است که در هر عملیات به تصویر کشیدن
\LTRfootnote{Render}
تنها یک ماده می‌تواند به تصویر کشیده شود. مثلا در همان مثال ماشین، قسمت‌های مختلف ماشین از ماده‌های مختلفی تشکیل می‌شود. به طور مثال چرخ ماشین می‌تواند از جنس آلومینیوم باشد یا لاستیک چرخ از جنس پلاستیک باشد و یا حتی قسمت‌های داخلی ماشین مانند صندلی ماشین از جنس چرم باشد.
بنابراین باید این قسمت‌ها به صورت جدا قرار گیرند تا بتوان هر قسمت را با توجه به ماده‌ی موردنظر آن به تصویر کشاند.

\section{ماده 
\protect\LTRfootnote{Material}
}
ماده‌ها شامل پارامتر‌های قابل تنظیمی هستند که با تنظیم آن به گرافیک ما اعلام می‌کند که چگونه باید یک مثلث را به تصویر بکشد.
این پارامترها می‌توانند شامل موارد زیر باشند ولی محدود به آن نمی‌شوند

\begin{enumerate}
	\item میزان کدورت و شفافیت شئ
 	\item میزان براقی شئ
 	\item رنگ(بافت) شئ
 	\item سایه‌زنی پیکسلی یا راسی \protect\LTRfootnote{Vertex or Pixel shader}
\end{enumerate}


\section{بافت
\protect\LTRfootnote{Texture}
}
بافت یک تصویر دوبعدی و یا سه‌بعدی است که می‌تواند در ماده استفاده شود.
این تصاویر به عنوان ورودی در برنامه دریافت شده و پس از اینکه یک شناسه به آن ها تخصیص داده شد، در کارت گرافیکی قرار می‌گیرند. ماده‌ها با استفاده از این شناسه می‌توانند در صورت لزوم به این بافت دستیابی پیدا کنند.

\chapter {آشنایی با موتور بازی}

\section{ساخت پروژه}

برای اینکه بتوان از موتور بازی‌سازی آنریل استفاده کرد ، ابتدا باید پروژه ای متناسب با کاری که می‌خواهیم انجام دهیم، بسازیم.
\\
برای ساختن پروژه به صورت زیر عمل می‌کنیم.
\\
ابتدا مرورگر آنریل انجین را باز می‌کنیم.


سپس با انتخاب دسته بندی مناسب و انتخاب گزینه‌ی بعدی به صفحه‌ی زیر هدایت می‌شویم. در اینجا  زیرمجموعه‌ی بازی را انتخاب شده است.

\begin{figure}[H]
	\centerline{\includegraphics[width=\textwidth,height=\textheight,keepaspectratio]{Figures/Ch2/UnrealEngineBrowser.png}}
	\caption{مرورگر پروژه‌های آنریل}
	\label{fig:Unreal engine Browser}
\end{figure}

\par\bigskip 
\noindent


در اینجا قالب مناسب را انتخاب کرده و گزینه‌ی بعدی را کلیک می‌کنیم.

\begin{figure}[H]
	%\centerline{\includegraphics[scale=0.5]{Figures/Ch2/UnrealEngineBasicClassesUML.png}}
	\centerline{\includegraphics[width=\textwidth,height=\textheight,keepaspectratio]{Figures/Ch2/TemplateSelection.png}}

	\caption{قالب پروژه‌های آنریل}
	\label{fig:Unreal engine Template}
  \end{figure}





در اینجا پس از انجام تنظیمات اولیه پروژه گزینه‌ی ایجاد پروژه را کلیک کرده و پروژه ساخته می‌شود.


\begin{figure}[H]
	\centerline{\includegraphics[width=\textwidth,height=\textheight,keepaspectratio]{Figures/Ch2/UnrealEngineProjectSetting.png}}
	\caption{تنظیم پروژه‌های آنریل}
	\label{fig:Unreal engine Setting}
\end{figure}


\section{ویرایشگر آنریل}
زمانی که پروژه ساخته می‌‌شود، ویرایشگر آنریل باز می‌شود. این ویرایشگر شامل پنل های مختلفی است که در عکس شماره زده شده و هر شماره نیز در جدول زیر توضیح داده شده است.

\begin{figure}[H]
	\centerline{\includegraphics[width=\textwidth,height=\textheight,keepaspectratio]{Figures/Ch2/UnrealEngineEditor.png}}
	\caption{ویرایشگر آنریل}
	\label{fig:Unreal engine Editor}
\end{figure}

\begin{table}[ht]
	\caption{مدلهای تبدیل.}
	\label{tab:MotionModels}
	\centering
	\onehalfspacing
	\resizebox{\textwidth}{!}{
	\begin{tabular}{|r|c|l|r|}
		\hline شماره & نام &  \makecell{توضیح} \\ 
		\hline 1 & نوار ابزار & \makecell{شامل توابع مختلفی است که به صورت معمول استفاده می‌شود}\\ 
		\hline 2 & بازیگران &   \makecell{می‌توان از این قسمت، بازیگر مناسب خود را انتخاب کرده و در صحنه‌ی بازی قرار داد}\\ 
		\hline 3 & درگاه دید
		\LTRfootnote{Viewport} & \makecell{از طریق این پنل می‌توان اشیاء را در محیط قرار داد و بر روی اشیاء قرار گرفته شده کلیک کرد.}  \\ 
		\hline 4 & طرح کلی جهان
		\LTRfootnote{World Outliner} &  \makecell{تمامی اشیاء در مرحله فعلی را نشان می‌دهد. \\
		 می‌توان اشیا را با قرار دادن آن‌ها در پوشا‌ها سازماندهی کرد. همچنین قابلیت جستجو بر اساس نوع را نیز دارد.}\\
		\hline 5 & مرورگر محتوا & \makecell{این پنل تمامی فایل‌های پروژه را نمایش می‌دهد. \\ همچنین می‌توان با ایجاد پوشه فایل‌ها را دسته بندی و مرتب کرد. \\ علاوه بر این با استفاده از نوار جستجو می‌توان فایل موردنظر خود را پیدا کرد}\\ 
		\hline 6 & جزئیات & \makecell{ این پنل برای نمایش یا تغییر ویژگی‌های شئ انتخاب شده است.\\
		 با تغییر ویژگی‌ها تنها ویژگی شی انتخاب شده تغییر پیدا می‌کند.} \\ 
		\hline 
	\end{tabular} }
\end{table}

\section{زبان برنامه‌نویسی}

موتور بازی‌سازی آنریل از زبان برنامه نویسی 
\lr{C++}
به همراه اسکریپ بصری به نام 
\lr{Blueprint}
استفاده می‌کند.

\lr{Blueprint}
یک سیستم برنامه‌نویسی کامل گیمپلی مبتنی بر مفهوم استفاده از رابط‌های مبتنی بر گره برای ایجاد عناصر گیمپلی از درون ویرایشگر است.
این سیستم بسیار منعطف و قدرتمند است زیرا این توانایی را در اختیار طراحان قرار می دهد تا از طیف گسترده ای از مفاهیم و ابزارها که عموماً فقط در دسترس برنامه نویسان هستند استفاده کنند.
\cite{UnrealEngineBlueprint}

\section{رایج‌ترین اصطلاحات}

در این بخش رایج‌ترین اصطلاحات مورد استفاده در هنگام کار با موتور بازی‌سازی آنریل را بررسی می‌کنیم.

\subsection{ پروژه }

یک پروژه آنریل، شامل تمامی محتوای بازی است. 
محتوا می تواند خود به چندین پوشه که بر روی دیسک قرار دارند، تقسیم شود.
بدیهی است که می توان این پوشه‌ها را به صورت دلخواه نامگذاری و سازماندهی کرد.
پنل مروگر محتوا داخل ویرایشگر آنریل همان ساختار راهنمای موجود در پوشه 
\lr{Project}
که بر روی دیسک قرار دارد را نشان می‌دهد.

هر پروژه دارای یک پرونده‌‌ی
\lr{.uproject}
متناظر به خود است.این فایل نحوه ایجاد، باز کردن یا ذخیره یک پروژه است. بنابراین می‌توان چندین پروژه مختلف ایجاد کرد و به صورت موازی بر روی آن‌ها کار کرد.

\subsection{
شئ
}

اشیاء پایه‌ای ترین کلاس موجود در آنریل هستند. آنها مانند بلوک‌های سازنده عمل می‌کنند و دارای بسیاری از عملکرد‌ها و توابع موردنیاز برای دارایی‌ها
\lr{Assets}
هستند.


\subsection{
بازیگران \protect\LTRfootnote{Actors}
}

هر شئ‌ای را که بتوان بر روی صحنه قرار داد مانند دوربین، مش استاتیک، محل شروع بازی و ... را بازیگر
\LTRfootnote{Actor}
می‌گویند.
بازیگران از تبدیل‌های سه‌بعدی مانند انتقال، دوران و تغییر مقیاس پشتیبانی می‌کنند.
آن‌ها را می‌توان از طریق کد‌های گیمپلی (
	\lr{C++}
	یا برنامه‌کار 
	\LTRfootnote{Blueprint}
)
ایجاد کرد و یا از بین برد.

\subsection{تغییر نوع داده 
\protect\LTRfootnote{Casting}}

تغییر نوع داده، عملی است که طی آن بازیگری از یک کلاس را گرفته و سعی می‌کند به گونه‌ای رفتار کند که گویی از کلاس دیگری است.
این عمل ممکن است موفقیت آمیز باشد و یا شکست بخورد. در صورت موفقیت آمیز بودن می‌توان به توابع کلاسی که به آن تغییر داده شده دستیابی پیدا کرد.


\chapter {سیستم گراف پویانمایی در موتور بازی‌سازی آنریل}

موتور آنریل مجموعه کاملی از ابزار‌های تولید محتوا برای توسعه‌ی بازی، مصورسازی معماری و خودرو، 
ایجاد محتوا برای فیلم و تلوزیون،
پخش رویداد‌های زنده، آموزش، شبیه‌سازی 
و ساید برنامه‌های بلادرنگ است.

این موتور برای اولین بار برای توسعه‌ی بازی "غیرواقعی" در سال 1998 توسعه پیدا کرد.
پس از آن نسخه‌های متعددی از این موتور منتشر شده است.
\cite{UnrealEngineWikiPedia}

موتور آنریل مانند تمامی موتور‌های بازی‌سازی دارای مولفه‌های فراوانی است که برای 
تولید بازی به کار می‌رود.
مولفه‌های پویانمایی، هوش مصنوعی، رندر، رابط کاربری تنها تعداد اندکی از مولفه‌هایی است که ‌می‌توان در 
آنریل استفاده کرد.

آنریل طیف گسترده‌ای از ابزار‌های قدرتمند را برای مدیریت شخصیت‌ها، ایجاد محتوای سینمایی 
و پویانمایی را ارائه می‌دهد.
با استفاده از سیستم پویانمایی مش اسکلتی، 
کاربران می‌توانند شخصیت‌ها، اسکلت‌ها و کلیپ‌های پویانمایی 
وارد شده خود را مدیریت کنند.
سپس این محتوا می‌تواند برای ایجاد گیم‌پلی تعاملی پویا شده با استفاده از 
ویژگی‌های مختلف مانند 
فضا‌های ترکیبی 
\LTRfootnote{Blend Spaces}
، طرح‌های پویانمایی 
\LTRfootnote{Animation Blueprint}
و ماشین‌های حالت 
\LTRfootnote{State Machines}
استفاده شود.
سکانس‌های سینمایی را می‌توان با استفاده از ابزار 
\lr{Sequencer}
ایجاد کرد. با استفاده از این ابزار می‌توان 
دوربین‌ها و شخصیت‌ها را متحرک ساخت.
پویانمایی شخصیت‌ها را می‌توان با استفاده از 
\lr{Control Rig}
که ابزار داخلی 
موتور آنریل است، انجام داد.
با استفاده از این ابزار می‌توان ریگ‌های مناسبی ساخت تا 
درون 
\lr{Sequencer}
شخصیت را متحرک ساخت.
\cite{UnrealEngineAnimation}

همانطور که مشخص است، سیستم پویانمایی موجود در موتور آنریل بسیار گسترده است. در این پروژه 
قسمت طرح پویانمایی آنریل که به گراف پویانمایی نیز شناخته می‌شود بررسی می‌شود.

برای اینکه بتوانیم در مورد سیسیتم گراف پویانمایی آنریل توضیح دهیم، ابتدا لازم است 
توضیحاتی را درباره‌ی نحوه‌ی معماری این انجین بیاوریم. بنابراین در این بخش ابتدا توضیح کوتاهی 
درباره‌ی معماری آنریل با محوریت نحوه‌ی رابطه‌‌ی اشیا با یکدیگر داده و 
پس از آن به بررسی ویژگی‌های سیستم گراف پویانمایی می‌پردازیم.

 
\section{بازیگران، پیاده‌ها و شخصیت‌‌ها}

اشیا در آنریل می‌توانند به سه کلاس کلی بازیگران
\LTRfootnote{Actors}
، پیاده‌ها 
\LTRfootnote{Pawns}
و شخصیت‌ها
\LTRfootnote{Characters}
دسته‌بندی می‌شوند.

بازیگران کلاس پایه‌ی تمامی اشیا ای هستند که به صورت فیزیکی می‌توانند در محیط سه‌بعدی قرار گیرند.
پیاده‌ها کلاسی مشتق شده از بازیگران هستند که بازیکنان می‌توانند کنترل آن‌ها را بدست گیرند و 
در محیط حرکت کنند.
 در نهایت شخصیت‌ها پیاده‌هایی هستند که دارای مش اسکلتونی، توانایی شناسایی برخورد و منطق حرکتی هستند.
 آنها مسئول تمام تعاملات فیزیکی بین بازیکن یا هوش مصنوعی، با جهان هستند و همچنین مدل های اولیه شبکه و دریافت ورودی را پیاده سازی می کنند. 
اگر بخواهیم شخصیت درون بازی از پویانمایی اسکلتونی استفاده کند، باید از این کلاس بهره ببریم.

\section{اجزاء}

اجزاء
\LTRfootnote{Components}
مجموعه‌ای از توابع و ویژگی‌ها است که می‌تواند به یک بازیگر اضافه شود.
بنابراین بازیگران می‌توانند حاوی مجموعه‌ای از
\lr{ActorComponents}
باشند که این اجزاء می‌توانند برای موارد مختلفی از جمله
کنترل نحوه‌ی حرکت بازیگران، 
نحوه‌ی رندر شدن و غیره استفاده شوند.

زمانی که یک مولفه به یک بازیگر اضافه می‌شود، آن بازیگر می‌تواند عملکرد‌های موجود در آن مولفه را استفاده کند.
به عنوان مثل یک مولفه نور نقطه‌ای باعث می‌شود که بازیگر مانند یک نور نقطه‌ای، نور ساطع کند.
یا یک مولفه صوتی به بازیگر این توانایی پخش صدا را می‌دهد.

مولفه‌ها حتما باید به یک بازیگر متصل شوند و به خودی خود نمی‌توانند وجود داشته باشند.
درواقع وقتی ما مولفه‌های مختلف را به بازیگر خود متصل می‌کنیم، در حال قرار دادن قطعه‌ها و تکه‌هایی هستیم
 که مجموع آن‌ها یک بازیگر را به عنوان یک موجودیت واحد که در محیط سه‌بعدی قرار می‌گیرد، تعریف می‌کنند.
 به عنوان مثال چرخ‌های یک ماشین، فرمان ماشین، چراغ‌ها و غیره همه به عنوان
 مولفه‌های ماشین درنظر گرفته می‌شوند در حالی که خود آن ماشین، بازیگر است.

\section{شخصیت‌ها}

هر شخصیت در آنریل از سه مولفه‌ی اصلی تشکیل شده است.


\begin{itemize}
	\item \lr{Skeletal Mesh Component}
	\item \lr{Character Movement Component}
	\item \lr{Capsule Component}
\end{itemize}


مولفه‌ی 
\lr{Skeletal mesh Component }
شامل طرح پویانمایی شخصیت است. طرح پویانمایی، سیستم پویانمایی شخصیت است که جلوتر آن را توضیح می‌دهیم.


مولفه‌ی 
\lr{ Character Movement Component}
همانطور که از اسمش مشخص است برای منطق حرکت در حالت‌های مختلف از جمله راه‌رفتن افتادن و غیره استفاده می‌شود.
این مولفه شامل تنظیمات و عملکرد‌های مربوطه برای کنترل حرکت است.

و در نهایت مولفه‌‌ی
\lr{Capsule Component}
وظیفه‌ی تشخیص برخورد در هنگام حرکت را دارد.


\section{مولفه‌ی مش اسکلتی}

این مولفه‌، مولفه‌ای است که به شخصیت امکان پویا شدن را می‌دهد.
این کلاس برای ساختن یک نمونه از کلاس 
\lr{SkeletalMesh}
است که بر روی آن کلیپ‌های پویانمایی اجرا می‌شوند.
اینکه چه کلیپ پویانمایی بر روی آن اجرا شود از طریق کلاس 
\lr{AnimInstance}
که همان طرح پویانمایی
\LTRfootnote{Animation Blueprint}
 است، انتخاب می‌شود.

همانطور که در فصل گذشته اشاره شد، مش اسکلتی 
شامل یک هندسه‌ی چندضلعی است که به یک اسکلت که در واقع 
سلسله مراتبی از مفاصل است، متصل است و این اسکلت می‌تواند به 
منظور تغییر شکل آن هندسه‌ی چندضلعی یا مش، متحرک شود.

مش‌های اسکلتی از دو قسمت ساخته‌شده‌اند. مجموعه‌ای از چندضلعی‌ها
که به منظور تشکیل سطح مش با یکدیگر ترکیب می‌شوند و 
یک اسکلت سلسله‌مراتبی که می‌تواند برای متحرک‌سازی چند‌ضلعی‌ها استفاده شود.
مدل‌های سه‌بعدی، 
اسکلت
و کلیپ‌های پویانمایی 
در یک برنامه مدل‌سازی و ایجاد پویانمایی
مانند 
\lr{Maya}،
\lr{3DSMax}
و ابزار‌های مدل‌سازی دیگر ایجاد می‌شوند.

در آنریل کلاس 
\lr{SkeletalMesh}
وظیفه‌ی نگهداری این مش اسکلتی را دارد.

همانگونه که گفتیم، نیاز داریم تا یک سیستمی داشته باشیم 
که بتواند کلیپ‌های پویانمایی را بر روی 
این مش اسکلتی اجرا کند. به زبانی دیگر، این 
مش اسکلی را پویا و متحرک سازد.
در آنریل کلاس 
\lr{AnimInstance}
وظیفه‌ی این عمل را دارد.

\section{طرح پویانمایی}

طرح پویانمایی یک طرح تخصصی است که پویانمایی 
یک مش اسکلتی را کنترل می‌کند.
با ویرایش گراف‌های موجود در این طرح، 
میتوان کار‌های مختلفی را روی پویانمایی شخصیت انجام داد.
به عنوان مثال می‌توان کلیپ‌های مختلف را با یکدیگر ترکیب کرد،
مستقیما مفاصل درون اسکلت را کنترل کرد و
یا هر تنظیمات منطقی‌ای که باعث تعریف 
ژست نهایی شخصیت در فریم فعلی می‌شود را انجام داد.

دو جزء اصلی در طرح پویانمایی وجود دارد که با هم کار می‌کنند تا ژست نهایی شخصیت را برای هر 
فریم ایجاد کنند.
این دو مولفه، گراف رویداد و گراف پویانمایی نام دارند.

به صورت کلی گراف رویداد مقادیری را که در گراف پویانمایی استفاده می‌شوند را به‌روز‌رسانی می‌‌کند تا 
در ماشین‌های حالت، فضاهای ترکیب و بقیه‌ی نود‌هایی که در گراف پویانمایی
 استفاده می‌شوند، به کار روند.

\section{گراف رویداد}

درون هر طرح پویانمایی، یک گراف رویداد وجود دارد. 
این گراف برای دریافت مقادیر منطقی از بخش گیمپلی و منطق بازی به‌کار می‌رود.
به عنوان مثال اینکه شخصیت می‌خواهد به چه سمتی حرکت کند، یا اینکه چه سرعتی دارد را 
از طریق این گراف در متغیرهایی که تعریف می‌کنیم، ذخیره می‌کنیم.
\cite{EventGraphUnrealEngine}

\begin{figure}[ht]
	\centerline{\includegraphics[width=\textwidth,height=8cm,keepaspectratio]{Figures/Ch3/EventGraph.png}}

	\caption{نمونه‌ای از یک گراف رویداد}
	\label{fig:EventGraph}
\end{figure}


\section{گراف پویا‌نمایی}

گراف پویانمایی برای ارزیابی ژست نهایی مش اسکلتی در فریم فعلی استفاده می‌شود.
به صورت کلی هر طرح پویانمایی دارای یک گراف پویانمایی است که 
این گراف شامل گره‌های مختلفی است که هر کدام از این گره‌ها استفاده‌های متفاوتی دارند.
به عنوان مثال می‌توان از این گره‌ها برای نمونه‌برداری


\begin{figure}[ht]
	\centerline{\includegraphics[width=\textwidth,height=8cm,keepaspectratio]{Figures/Ch3/AnimationGraph.png}}

	\caption{نمونه‌ای از یک گراف پویانمایی}
	\label{fig:AnimationGraph}
\end{figure}





\LTRfootnote{Sampling}
از دنباله‌های کلیپ‌های پویانمایی،
انجام ترکیب‌های بین کلیپ‌ها 
یا کنترل تبدیل‌های مربوط به مفاصل استفاده کرد.
سرانجام ژست نهایی بدست آمده روی 
مش اسکلتی در پایان هر فریم اعمال می‌شود.

\section{گره‌های گراف پویانمایی }

گراف پویا‌نمایی با ارزیابی گره‌های موجود در گراف، عمل می‌کند.
بعضی از گره‌های موجود در گراف، عملیات‌های خاصی را بر روی یک یا چند
ژست ورودی انجام می‌دهند، در حالی که برخی دیگر 
برای دسترسی یا نمونه‌برداری از انوع دیگری از دارایی‌ها مانند
فضاهای ترکیب
\LTRfootnote{Blend Spaces}
، مونتاژهای پویانمایی
\LTRfootnote{Animation Montages}
و دنباله‌های پویا‌نمایی
\LTRfootnote{Animation Sequence}
استفاده می‌شوند.
ماشین‌های حالت نیز که حاوی شبکه‌ی نموداری خودشان هستند، می‌توانند 
به صورت تنهایی یا با ترکیب با یکدیگر در 
گراف پویانمایی استفاده شوند.

در این بخش ابتدا با نحوه‌‌ی جریان اجرا در گراف پویانمایی 
آشنا می‌شویم، سپس ساختار کلی گره‌های موجود را بررسی کرده و در نهایت 
به بررسی انواع گره‌‌های موجود در گراف پویانمایی می‌پردازیم.



\section{ساختار گره‌های گراف پویانمایی}

گره‌‌ها می‌توانند شامل چند پین ورودی که درواقع ژست‌های ورود هستند، باشند.
به صورت کلی پین‌ها شامل یک خروجی هستند که این خروجی نشان‌دهنده‌ی 
ژست شخصیت پس از انجام عملیات‌های مربوط به آن پین است.
همچنین می‌توانند شامل پین‌های ویژگی باشند. مقادیر این پین‌های ویژگی از 
متغیر‌هایی که در گراف رویداد تعریف و مقداردهی شده‌اند، می‌توانند بدست آیند.

\begin{figure}[ht]
	\centerline{\includegraphics[width=\textwidth,height=8cm,keepaspectratio]{Figures/Ch3/AnimNodeStructure.png}}

	\caption{ساختار کلی نود‌های موجود در گراف پویانمایی}
	\label{fig:AnimNodeStructure}
\end{figure}

قابل ذکر است با انتخاب هر نود می‌توان به پنل جزئیات آن نود هم دسترسی پیدا کرد 
که به وسیله‌ی آن می‌توان تنظیمات لازم را بر روی آن نود انجام داد.

\begin{figure}[ht]
	\centerline{\includegraphics[width=\textwidth,height=8cm,keepaspectratio]{Figures/Ch3/DetailsPanel.png}}

	\caption{پنل تنظیمات گره‌ی ترکیب}
	\label{fig:AnimNodeDetailPanel}
\end{figure}

\section{جریان اجرا در گراف پویانمایی}

تمامی گراف‌ها دارای یک جریان اجرا هستند که به صورت پیوند‌های ضربانی 
میان ورودی و خروجی پین‌ها قابل مشاهده هستند. این جریان‌ها درواقع نحوه‌ی حرکت داده
را در گراف ترسیم می‌کنند.
در گراف پویانمایی، این جریان نشان‌دهنده‌ی ‌ژست‌هایی است که از 
یک گره به گره‌ی دیگر منتقل می‌شود.
در برخی از گره‌‌ها مانند گره‌ی ترکیب، ورودی‌های متعددی وجود دارند و به صورت 
درونی با مقادیری که در متغیر‌ها داریم تصمیم می‌گیرند که کدام یک از ورودی‌‌ها 
در حال حاضر بخشی از جریان اجرا است.


\begin{figure}[ht]
	\centerline{\includegraphics[width=\textwidth,height=8cm,keepaspectratio]{Figures/Ch3/ExecFlow.png}}

	\caption{نمونه‌ای از جریان اجرا}
	\label{fig:ExecFlow}
\end{figure}



\cite{AnimationGraphUnrealEngine}

\section {دنباله‌ی پویانمایی}

کلیپ‌های پویانمایی در آنریل به اسم دنباله‌ی پویانمایی شناخته می‌شوند.
دنباله‌ی پویانمایی یک دارایی پویانمایی است که 
حاوی داده‌‌های پویانمایی است که می‌تواند روی 
یک مش اسکلتی پخش شود تا شخصیت مربوط 
به آن اسکلت را متحرک سازد.
یک دنباله‌ی پویانمایی شامل فریم‌های کلیدی هستند که 
این فریم‌های کلیدی بیانگر موقعیت
\LTRfootnote{Position}
، دوران 
\LTRfootnote{Rotation}
، مقیاس
\LTRfootnote{Scale}
اسکلت مش در نقطه‌‌ی خاصی از زمان است.
بنابراین کلیپ‌های پویانمایی یکی از گره‌های مهم در 
گراف انیمشن حساب می‌آیند.

قابل ذکر است این گره‌ مانند بقیه‌ی گره‌ها دارای تنظیماتی است که در پنل تنظیمات قابل مشاهده‌ هستند. به عنوان مثال 
می‌توان سرعت حرکت کلیپ را در این تنظیمات مشخص کرد یا اینکه کلیپ به صورت حلقه‌وار تکرار شود یا خیر.

\begin{figure}[ht]
	\centerline{\includegraphics[width=\textwidth,height=8cm,keepaspectratio]{Figures/Ch3/AnimationSequence.png}}

	\caption{نمونه‌ای از گره‌ی دنباله‌ی پویانمایی}
	\label{fig:AnimationSequence}
\end{figure}



\section {فضای ترکیب}

فضای ترکیب یک دارایی به خصوص است که 
امکان ترکیب کلیپ‌های پویانمایی بر اساس مقدار مختلف ورودی را دارد.

به عنوان مثال فرض کنیم که چند کلیپ داریم که 
وضعیت حرکت شخصیت را مشخص می‌کنند.
این کلیپ‌ها می‌توانند به 
حالت ایستاده، حرکت با سرعت کم، حرکت با سرعت متوسط 
و حرکت با سرعت سریع تقسیم شوند.
علاوه‌ بر این‌ها می‌توان این پویانمایی‌ها را برای جهت‌های مختلف داشت.
در این صورت می‌توان یک فضای ترکیب داشت که بر اساس 
دو ورودی، سرعت و جهت عمل می‌کند.
و می‌توان هرکدام از این کلیپ‌ها را در جای مشخص خودش 
(با توجه به سرعت و جهت)
قرار داد.

درجلوتر اشاره می‌شود که می‌توان کلیپ‌ها را با استفاده از 
نود ترکیب نیز با یکدیگر ترکیب نمود. 
فضای ترکیب، ابزاری برای انجام ترکیب‌های پیچیده‌تر 
بین کلیپ‌های پویانمایی متعدد بر اساس 
مقادیر متفاوت است.
هدف این گره، کاهش نیاز به ایجاد گره‌های منفرد
در هنگامی که می‌خواهیم ترکیب بر اساس ویژگی‌ها یا شرایط خاصی 
صورت گیرد، است.


\begin{figure}[ht]
	\centerline{\includegraphics[width=\textwidth,height=8cm,keepaspectratio]{Figures/Ch3/BlendSpace.png}}

	\caption{در این فضای حالت، سرعت بین 0 تا 500 متغیر است و زاویه بین -120 تا 120 متغیر است.(در شکل به صورت کمی نوشته‌شده‌است)
	در کل از 10 کلیپ پویانمایی شده (نقاط سفیدرنگ در تصویر) ولی با استفاده از ترکیب بین این کلیپ‌ها می‌توان تمامی مقادیر بین کلیپ‌ها را نیز بدست آورد. }
	\label{fig:BlendSpace}
\end{figure}


\section {گره‌های ترکیب}

گره‌های ترکیب برای ترکیب چندین کلیپ پویانمایی با یکدیگر استفاده‌ می‌شوند.
این گره‌ تنها مختص گراف پویانمایی هستند و در گراف‌های دیگر مانند گراف رویداد 
نمی‌توان از آن‌ها استفاده کرد.
به صورت کلی هر کدام از این نوع گره‌ها دارای چند پین ورودی و 
یک آلفا یا وزن است که برای محاسبه‌ی 
وزن هرکدام از ژست‌های ورودی در ژست خروجی به کار می‌رود.
بعضی از این گره‌‌ها می‌توانند 
پیچیده‌تر نیز باشند و نیاز به داده‌های بیشتری به عنوان ورودی باشند.

در ادامه به بررسی چند نمونه از گره‌‌های ترکیب می‌پردازیم.

\subsection{ گره‌ی ترکیب استاندارد}

این گره برای ترکیب‌کردن دو ژست ورودی با گرفتن یک آلفا عمل می‌کند.
اگر ژست‌های ورودی را
\lr{A}
و 
\lr{B}
 و
خروجی نهایی را 
\lr{Output}
در نظر بگیریم، 
خروجی به صورت زیر محاسبه می‌شود.

\begin{equation}\label{eq:BlendNode}
	Output= A * (1-alpha) + B * alpha
\end{equation}

\cite{BlendNodeUnrealEngine}

\begin{figure}[ht]
	\centerline{\includegraphics[width=\textwidth,height=8cm,keepaspectratio]{Figures/Ch3/BlendNode.png}}\hfill
	\caption{ گره‌ی ترکیب }
	\label{fig:BlendNode}
\end{figure}

\subsection{ گره‌ی ترکیب بر اساس یک مقدار}

این نوع از گره‌ها برای ترکیب کلیپ‌های پویانمایی بر اساس یک مقدار 
که این مقدار می‌تواند عدد صحیح، مقدار بولین و یا مقداری از 
نوع داده‌ی 
\lr{Enum}
باشد.

به عنوان مثال فرض کنیم شخصیت درون بازی می‌تواند در وضعیت‌های مختلفی از نظر
حالت مبارزه با اسلحه قرار گیرد. این حالت‌ها می‌توانند حالت بدون اسلحه، 
حالت با اسلحه و حالت گرفتن نشانه با اسلحه باشد.
می‌توان این حالت‌ها را با
\lr{enum}
نشان داد.
اگر برای هرکدام از این حالات یک کلیپ پویانمایی داشته باشیم و بخواهیم 
با توجه به حالت فعلی شخصیت یکی از این کلیپ‌ها را روی شخصیت پخش کنیم،
می‌توانیم از این گره استفاده کنیم.
\cite{BlendNodeUnrealEngine}

این مثال در شکل زیر آمده است.

\begin{figure}[ht]
	\centerline{\includegraphics[width=\textwidth,height=8cm,keepaspectratio]{Figures/Ch3/EnumBlendNode_WithArrow.png}}\hfill
	\caption{ مثال استفاده از گره‌ی ترکیب }
	\label{fig:EnumBlendNode_WithArrow}
\end{figure}

\subsection{ گره‌ی ترکیب لایه‌ای برای هر مفصل}

با استفاده از این گره‌ می‌توانیم کلیپ‌ پویانمایی را بر روی 
مفاصل محدودی از اسکلت اجرا کنیم.
به عنوان مثال فرض کنید یک کلیپ پویانمایی مربوط به مشت زدن و کلیپ 
دیگری مربوط به حرکت شخصیت داریم. 
اگر بخواهیم از این دو کلیپ استفاده کنیم تا شخصیت در حال 
حرکت بتواند مشت هم بزند، می‌توان از این گره استفاده کرد.
با استفاده از این گره، می‌توان کلیپ مربوط به مشت زدن را تنها 
بر روی قسمت کمر به بالا‌‌‌ی شخصیت و کلیپ حرکت را بر 
روی کمر به پایین شخصیت اجرا کنیم.

\begin{figure}[ht]
	\centerline{\includegraphics[width=\textwidth,height=8cm,keepaspectratio]{Figures/Ch3/LayeredBlendPerBone_WithArrow.png}}\hfill
	\caption{ اجرای کلیپ‌‌‌های پویانمایی بر روی مفاصل متفاوت }
	\label{fig:LayeredBlendPerBone_WithArrow}
\end{figure}

\cite{BlendNodeUnrealEngine}


\section{گره‌های کنترل اسکلت}

این گره‌ها امکان دستکاری مستقیم مفاصل موجود در اسکلت را فراهم می‌کنند.
مجموعه‌ی این گره‌ها شامل حل‌کننده‌های مختلف هستند که می‌توان به 
حل کننده‌ی 
\lr{IK}
به عنوان مثال اشاره کرد.
علاوه بر این بعضی از گره‌های مربوط به این قسمت، می‌توانند برای اعمال فیزیک بر روی 
مفاصل استفاده شوند.
\cite{SkeletalControlsUnrealEngine}

\section{گره‌های تبدیل فضا}

در موتور آنریل ژست‌ها می‌توانند در فضای محلی یا فضای مولفه قرار گیرند.
در فضای محلی مفاصل نسبت به والد خود قرار می‌گیرند، در صورتی که 
در فضای مولفه، مفاصل نسبت به مولفه‌ی مش اسکلتی 
\LTRfootnote{SkeletalMeshComponent}
قرار می‌گیرند.
اکثر گره‌ها با ژست‌ها در هنگامی که در فضای محلی قرار دارند، کار می‌کنند.
اما بعضی از گره‌های ترکیب و تمامی گره‌های کنترل اسکلت 
با ژست در فضای مولفه کار می‌کنند. بنابراین لازم از در مواقع لازم با استفاده از 
این نود، ژست اسکلت را از یک فضا به فضای دیگر منتقل کنیم.
\cite{ChangeSpaceUnrealEngine}

\begin{figure}[ht]
	\centerline{\includegraphics[width=\textwidth,height=8cm,keepaspectratio]{Figures/Ch3/ChnageSpaceNodes.png}}\hfill
	\caption{ گره‌های تبدیل حالت }
	\label{fig:ChnageSpaceNodes}
\end{figure}

\section{گره‌‌ی ماشین حالت}

ماشین‌های حالت، یک راه گرافیکی برای شکستن پویانمایی 
شخصیت‌ها به یک سری حالت را ارائه می‌دهند.
در آنریل می‌توان ماشین‌های حالت پیچیده و تو در تو بر اساس نیاز کاربران به وجود آورد.


\begin{figure}[ht]
	\centerline{\includegraphics[width=\textwidth,height=8cm,keepaspectratio]{Figures/Ch3/AnimationStateMachine1.png}}\hfill
	\caption{ ماشین حالت برای حرکت شخصیت }
	\label{fig:AnimationStateMachine1}
\end{figure}

\begin{figure}[ht]
	\centerline{\includegraphics[width=\textwidth,height=8cm,keepaspectratio]{Figures/Ch3/AnimationStateMachine3.png}}\hfill
	\caption{ شخصیت ماشین حالت برای حرکت شخصیت در هنگام روی زمین قرار گرفتن  }
	\label{fig:AnimationStateMachine3}
\end{figure}

\section{نتیجه‌گیری}

در این فصل نگاهی بر موتور بازی‌سازی آنریل با تاکید بر گراف پویانمایی انداختیم.
همچنین توضیحات کاملی، درباره‌ی ویژگی‌هایی که گراف پویانمایی در اختیار کابران می‌گذارد، آورده شد.
بنابراین آشنایی کافی با ابزار‌هایی که یک سیستم پویانمایی در اختیار کاربران می‌گذارد و 
اینکه این ابزارها در چه مواردی استفاده می‌شوند، پیدا کردیم.
حال برای درک بیشتر آنچه در داخل این سیسیتم‌ها اتفاث می‌افتد، به پیاده‌سازی یک سیستم پویانمایی از پایه می‌پردازیم.




\chapter { پیاده سازی }

در این بخش به روش‌ها و ابزارهای استفاده شده در پیاده‌سازی سیستم انیمیشن اشاره خواهد شد

\section {ابزارها}

\subsection {
    \lr{OpenGL}
    }

\lr{OpenGL}
یک واسط برنامه نویسی کاربردی  
\LTRfootnote{API}
است که با فراهم کردن توابع زیادی به توسعه‌دهندگان امکان دستکاری گرافیک و تصاویر را می‌دهد.
\lr{OpenGL} 
یک کتابخانه‌ی رندرینگ است.
یک "شئ" به خودی خود در
\lr{OpenGL} 
مفهومی ندارد
و به صورت مجموعه‌ای از مثلث‌ها و حالات مختلف درنظر گرفته می‌شود. بنابراین  
وظیفه‌ی ما است که بدانیم چه شئ‌ای در کدام قسمت صفحه رندر شده است. این کتابخانه تنها وظیفه‌اش کشیدن تصاویری که است که می‌خواهیم به تصویر کشیده‌شوند.
در این صورت اگر می‌خواهیم تصویری را به‌روزرسانی کنیم و یا به عنوان مثال شئ‌ای را تحرک دهیم باید به 
\lr{OpenGL}
درخواست دهیم که صحنه را دوباره‌برای ما رندر کند.
\cite{KhronosUsingOpenGL}

به صورت کلی 
\lr{OpenGL}
را می‌توان یک ماشین حالت بزرگ درنظر گرفت. هر حالت شامل مجموعه‌ای از متغیر‌ها است که نحوه‌ی عملکرد
\lr{OpenGL}
را مشخص می‌کند. 
به مجموعه‌ی این حالت‌ها 
\lr{OpenGL context}
نیز می‌گویند. 
در واقع  
\lr{context}
را می‌توان یک شئ درنظر گرفت که کل
\lr{OpenGL}
را دربر می‌گیرد. عموما تمامی تغییرات روی 
\lr{context}
فعلی اعمال می‌شود و سپس رندر می‌شود.
\cite{KhronosUsingOpenGL} \cite{LearnOpenGL_GettingStarted}

%%%%%%%%%%%%%%%%%%%%%%%%%%%%%%%%%%%%%%%%%%%%%%%%%%%%%

\subsection{\lr{GLFW}}

از آنجایی که به‌وجود‌آوردن یک پنجره‌ی جدید و همچنین 
\lr{context}
وابسته به نوع سیستم‌عامل است بنابراین نیازمند کتابخانه‌ای هستیم که بتواند این موارد را برای ما مدیریت کند.
\lr{GLFW}
یک کتابخانه‌ی منبع باز و چندپلتفرمی برای 
\lr{OpenGL}
است که یک
\lr{API}
ساده و مستقل از پلتفرم برای تولید پنجره‌ها، زمینه‌‌ها
\LTRfootnote{Contexts}
و سطوح، خواندن ورودی و مدیریت رویداد‌ها
\LTRfootnote{Events}
ارائه می‌کند. 
این کتابخانه از سیستم‌عامل‌های 
ویندوز
و 
مک
و 
لینوکس
و سیستم‌های مشابه یونیکس پشتیبانی ‌می‌کند.
\cite{GLFW}


\subsection{\lr{GLAD}}
کتابخانه‌های گرافیکی مانند
\lr{OpenGL}
وظیفه‌‌ی پیاده‌سازی توابع گرافیکی را ندارند بلکه می‌توان آن را مانند یک هدر در زبان 
برنامه‌نویسی 
\lr{C++}
دانست که تعریف اولیه توابع را دارند. پیاده‌سازی این توابع در درایور‌های 
\lr{GPU}
قرار دارند.
دسترسی به این اشاره‌گر‌‌های تابع به خودی خود سخت نیست ولی از آنجایی که این اشاره‌گر ها وابسته به پلتفرم هستند بنابراین کار طاقت فرسایی است. 
وظیفه‌ی کتابخانه‌ی 
\lr{GLAD}
فراهم سازی و کنترل این اشاره‌گرهای تابع است.
\cite{GLAD}


\subsection{\lr{GLM}}
\lr{GLM}
یک کتابخانه‌ی ریاضی برای نرم‌افزارهای گرافیکی مبتنی بر زبان برنامه‌نویسی سایه‌ی 
\lr{OpenGL}
\LTRfootnote{OpenGL Shading Language(GLSL)} 
است. این کتابخانه تنها شامل یک هدر 
\lr{C++}
است.
توابع و کلاس‌های موجود در این کتابخانه به صورتی نامگذاری و طراحی شده‌آند که بسیار به 
\lr{GLSL}
 نزدیک باشند.


 \subsection{\lr{Assimp}}

 \lr{Assimp}
 یک کتابخانه برای بارگذاری و پردازش صحنه‌های هندسی از فرمت‌های مختلف است.
 می‌توان با استفاده از آن مواردی همچون مش‌های استاتیک و یا اسکلتونی، مواد 
 \LTRfootnote{Materials}
 ، انیمیشن های اسکلتونی و داده‌‌های بافت را از فایل بارگذاری کرد.
زمانی که این مدل‌ها بارگذاری می‌شوند این کتابخانه آن‌ها را در ساختاری به شکل زیر ذخیره می‌کند و بعد از آن می‌توان از این ساختار، داده‌های مورد نظر خود را خواند و از آن‌ها استفاده کرد.
\cite{Assimp} \cite{LearnOpenGL_Assimp}

\begin{figure}[ht]
	\centerline{\includegraphics[width=\textwidth,height=\textheight,keepaspectratio]{Figures/Ch5/assimp_structure.png}}

	\caption{ساختار کلاس‌های کتابخانه‌ی \lr{Assimp} \cite{LearnOpenGL_Assimp}}
	\label{fig:Assimp}
  \end{figure}
  


  
\subsection{\lr{stb}}

این کتابخانه برای بارگذاری تصاویر استفاده می‌شود. در این پروژه از این کتابخانه برای بارگذاری
تصاویر بافت‌ها در کنار کتابخانه‌ی 
\lr{Assimp}
استفاده شده است.
\cite{stb}


\section{پیاده‌سازی}

این بخش دو هدف کلی را دنبال می‌کند.

\begin{enumerate}
	\item نمایش مدل گرافیکی و اجرای انیمیشن بر روی ‌آن
	\item ترکیب انیمیشن‌های مختلف به وسیله‌ی ماشین حالت

\end{enumerate}

\subsection{نمایش مدل گرافیکی}

همانطور که گفته شد مدل‌ها یا اشیاء سه‌بعدی به خودی خود مفهومی در 
\lr{OpenGL}
ندارند. آنچه برای 
\lr{OpenGL}
اهمیت دارد لیستی از مثلث‌ها است تا آن‌ها را به تصویر بکشد.
مدل‌های سه‌بعدی از رئوس، لبه و وجوه تشکیل می‌شوند و در فرمت‌های مختلفی مانند
\lr{FBX}
ذخیره می‌شوند. در این پیا‌ده‌سازی، از کتابخانه‌ی 
\lr{Assimp}
برای خواندن این داده‌ها استفاده شده است.

\subsection{قرارگیری مدل سه‌بعدی در کارت گرافیک}

همانطور که اشاره‌شد آنچه برای کارت گرافیک اهمیت دارد این است که به آن مجموعه‌ای از مثلث‌ها داده شود تا برایمان ترسیم کند.
برای اینکار به صورت عمومی از 3 آرایه مختلف استفاده می‌شود که به نام‌های 
\lr{VBO}
،
\lr{VAO}
و 
\lr{EBO}
شناخته می‌شوند.
\lr{VBOs}
\LTRfootnote{Vertex Buffer Objectss}
یک آرایه یا بافری است که تمامی رئوس مدل سه‌بعدی ما را در خود جای می‌دهد.
همانطور که در بخش  2-2-1
اشاره شد، رئوس علاوه بر اینکه شامل اطلاعات موقعیت مکانی در محیط سه‌بعدی هستند، شامل اطلاعات دیگری 
همچون رنگ، بردار نرمال، مختصات بافت  و... نیز می‌توانند باشند. بنابراین باید به صورتی به کارت گرافیک 
اعلام کنیم که این داده‌ای که در آرایه‌ی 
\lr{VBOs}
قرار دارد را چگونه تفسیر کند.
اینکار با استفاده از یک آرایه‌ی دیگر به نام 
\lr{VAO}
\LTRfootnote{Vertex Array Objects}
صورت می‌گیرد.
در نهایت گفتیم که آنچه برای کارت گرافیک اهمیت دارد دریافت مثلث‌ها است. بنابراین باید به طریقی بگوییم کدارم رئوس با 
اتصال به یکدیگر مثلث تشکیل می‌دهند. اینکار نیز با استفاده از آرایه‌ی 
\lr{EBOs}
\LTRfootnote{Element Buffer Objects}
صورت می‌گیرد.

%\input{Chapters/Chaptertemplate}


% ░░░░░░░▒▒▒▒▒▒▓▓▓▓ Appendices ▓▓▓▓▒▒▒▒▒▒░░░░░░░
\MakeAppendices%
%\input{Chapters/Appendices}

% ░░░░░░░▒▒▒▒▒▒▓▓▓▓ References ▓▓▓▓▒▒▒▒▒▒░░░░░░░
\MakeReferences%
\bibliographystyle{Settings/ModifiedIEEEtranFa}%
\bibliography{References}%

% ░░░░░░░▒▒▒▒▒▒▓▓▓▓ Abstract - English ▓▓▓▓▒▒▒▒▒▒░░░░░░░
\DepartmentEn{Department of Electrical and Computer Engineering}
\DegreeEn{Bachelor of Science}%\DegreeEn{Master of Science (MSc)} % Or \DegreeEn{Doctor of Philosophy (PhD)}
\YourFullnameEn{Nami Naziri}
\YourEmailAddress{nami.naziri@yahoo.com}
\DateEn{May 22, 2022}
\FirstSupervisorEn{Maziar Palhang, Assoc. Prof.}
\FirstSupervisorEmailAddress{palhang@cc.iut.ac.ir}
\TitleEn{Analysis of the animation graph in Unreal Engine and \\[0.2cm] implementation of an animation system using OpenGL }
% اگر عنوان رساله طولانی بود، در دو خط به صورت نشان داده شده تقسیم شود.

%todo\input{Chapters/AbstractEn}%
\MakeEnglishAbstract%

% ░░░░░░░▒▒▒▒▒▒▓▓▓▓ Signature - English ▓▓▓▓▒▒▒▒▒▒░░░░░░░
%\FirstAdvisorEn{First Advisor, Assoc. Prof.}
%\SecondAdvisorEn{Second Advisor, Assist. Prof.} % Optional (Remove It If You Don't Have)
\FirstExaminerEn{First Examiner, Assoc. Prof.} % Optional (Remove It If You Don't Have)
\SecondExaminerEn{First Examiner, Assist. Prof.} % Optional (Remove It If You Don't Have)
%\ThirdExaminerEn{Third Examiner, Prof.} % Optional (Remove It If You Don't Have)
%\FourthExaminerEn{Fourth Examiner, Prof.} % Optional (Remove It If You Don't Have)
%\FifthExaminerEn{Fifth Examiner, Prof.} % Optional (Remove It If You Don't Have)
\DeanOfDepartmentEn{Reza Tikani, Assist. Prof.}

\MakeEnglishSignaturePage%

\end{document} 

