\AbstractFa{
    پویانمایی کامپیوتری فرایندی است که برای تولید تصاویر متحرک دیجیتالی استفاده می‌شود. پویانمایی کامپیوتری مدرن معمولا از گرافیک کامپیوتری سه‌بعدی برای ایجاد یک تصویر سه‌بعدی استفاده می‌کند. در اکثر سیستم‌های پویانمایی کامپیوتری سه‌بعدی یک پویاساز نمایش ساده از آناتومی یک شخصیت ایجاد می‌کند که مشابه یک اسکلت یا آدمک است. در شخصیت‌های انسان و حیوانات اکثر قسمت‌های این مدل اسکلتی با استخوان‌های واقعی مطابقت دارد.
    \\
    گام اول این پروژه بررسی سیستم گراف پویانمایی در موتور بازی‌سازی آنریل است. گراف پویانمایی برای محاسبه‌ی وضعیت نهایی یک مش اسکلتی در فریم فعلی استفاده می‌شود. به صورت کلی این گراف برای نمونه‌گیری ، ترکیب و دستکاری ژست‌ها استفاده می‌شود و این ژست به مش‌های اسکلتی توسط طرح پویانمایی اعمال می‌شود. در این گام به بررسی این گراف و الگوریتم‌های به کار ‌گرفته ‌شده در آن خواهیم‌ پرداخت. 
    \\
    در گام دوم نیز به پیاده‌سازی سیستم پویانمایی اسکلتی از پایه، با توجه به روش‌های بدست‌آمده پرداخته می‌شود. این مرحله سه هدف را دنبال می‌کند. هدف اول نمایش اسکلتون در یک محیط سه‌بعدی و نورپردازی آن، که با استفاده از 
    \lr{OpenGL}
    به ‌وجود آمده، است. در این مرحله باید با استفاده از زبان 
    \lr{C++} 
    برنامه‌ای بنویسیم که در نهایت بتواند یک کلیپ پویانمایی به‌ وجود آمده به وسیله‌ی فریم‌های کلیدی را نمایش دهد. هدف دوم اضافه کردن یک مش به اسکلتون با استفاده از روش های پوسته سازی است. در هدف نهایی نیز روش ترکیب کلیپ‌های پویانمایی را پیاده‌سازی می‌کنیم تا بتوانیم کلیپ‌های مختلف را با یکدیگر ترکیب کنیم. 
}

\KeywordsFa{
 پویانمایی کامپیوتری،  موتور بازی‌سازی ، موتور آنریل،  گراف پویانمایی، گرافیک سه‌بعدی کامپیوتری
}