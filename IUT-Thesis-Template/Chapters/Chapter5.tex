\chapter { پیاده سازی }



برای پیاده‌سازی سیستم پویانمایی از 
\lr{OpenGL}
که یک واسط برنامه‌نویسی کاربردی برای دسترسی به توابع گرافیکی است استفاده شده است.
در این فصل ابتدا به بررسی کتابخانه‌های کمکی استفاده شده در این پیاده‌سازی می‌پردازیم. سپس به 
نحوه‌ی نمایش مدل‌های گرافیکی می‌پردازیم و پس از آن به بررسی پیاده‌سازی 
سایه‌زنی فونگ و فرمول‌های مرتبط با آن می‌پردازیم. در نهایت نحوه‌ی پیاده‌سازی سیستم پویانمایی و 
الگوریتم‌های به کار رفته در آن بررسی می‌‌شوند.

\section {کتابخانه‌‌های کمکی}

در این پیاده‌سازی از کتابخانه‌‌های مختلفی استفاده شده‌اند. هرکدام از این کتابخانه‌ها وظیفه‌ی 
به خصوص خود را دارند. در این بخش به معرفی و بررسی این کتابخانه‌ها می‌پردازیم.


\subsection{\lr{GLFW}}

از آنجایی که به‌وجود‌آوردن یک پنجره‌ی جدید و همچنین 
\lr{context}
وابسته به نوع سیستم‌عامل است بنابراین نیازمند کتابخانه‌ای هستیم که بتواند این موارد را برای ما مدیریت کند.
\lr{GLFW}
یک کتابخانه‌ی منبع باز و چندپلتفرمی برای 
\lr{OpenGL}
است که یک
\lr{API}
ساده و مستقل از پلتفرم برای تولید پنجره‌ها، زمینه‌‌ها
\LTRfootnote{Contexts}
و سطوح، خواندن ورودی و مدیریت رویداد‌ها
\LTRfootnote{Events}
را ارائه می‌کند. 
این کتابخانه از سیستم‌عامل‌های 
ویندوز
، 
مک
و 
لینوکس
و سیستم‌های مشابه یونیکس پشتیبانی ‌می‌کند.
\cite{GLFW}


\subsection{\lr{GLAD}}
کتابخانه‌های گرافیکی مانند
\lr{OpenGL}
وظیفه‌‌ی پیاده‌سازی توابع گرافیکی را ندارند بلکه می‌توان آن‌ها را مانند یک هدر در زبان 
برنامه‌نویسی 
\lr{C++}
دانست که تعریف اولیه توابع را دارند. پیاده‌سازی این توابع در درایور‌های 
\lr{GPU}
قرار دارند.
دسترسی به این اشاره‌گر‌‌های تابع به خودی خود سخت نیست ولی از آنجایی که این اشاره‌گرها وابسته به پلتفرم هستند بنابراین کار طاقت فرسایی است. 
وظیفه‌ی کتابخانه‌ی 
\lr{GLAD}
فراهم سازی و کنترل این اشاره‌گرهای تابع است.
\cite{GLAD}


\subsection{\lr{GLM}}
\lr{GLM}
یک کتابخانه‌ی ریاضی برای نرم‌افزارهای گرافیکی مبتنی بر زبان برنامه‌نویسی سایه‌ی 
\lr{OpenGL}
\LTRfootnote{OpenGL Shading Language(GLSL)} 
است. این کتابخانه تنها شامل یک هدر 
\lr{C++}
است.
توابع و کلاس‌های موجود در این کتابخانه به صورتی نامگذاری و طراحی شده‌آند که بسیار به 
\lr{GLSL}
 نزدیک باشند.


 \subsection{\lr{Assimp}}

 \lr{Assimp}
 یک کتابخانه برای بارگذاری و پردازش صحنه‌های هندسی از فرمت‌های مختلف است.
 می‌توان با استفاده از آن مواردی همچون مش‌های استاتیک و یا اسکلتونی، مواد 
 \LTRfootnote{Materials}
 ، کلیپ‌های پویانمایی اسکلتونی و داده‌‌های بافت را از فایل بارگذاری کرد.
زمانی که این مدل‌ها بارگذاری می‌شوند این کتابخانه آن‌ها را در ساختاری به شکل
\ref{fig:Assimp}
 ذخیره می‌کند و بعد از آن می‌توان از این ساختار، داده‌های مورد نظر خود را خواند و از آن‌ها استفاده کرد.
\cite{Assimp} \cite{LearnOpenGL_Assimp}

\begin{figure}[ht]
	\centerline{\includegraphics[width=\textwidth,height=\textheight,keepaspectratio]{Figures/Ch5/assimp_structure.png}}

	\caption{ساختار کلاس‌های کتابخانه‌ی \lr{Assimp} \cite{LearnOpenGL_Assimp}}
	\label{fig:Assimp}
  \end{figure}
  


  
\subsection{\lr{stb}}

این کتابخانه برای بارگذاری تصاویر استفاده می‌شود. در این پروژه از این کتابخانه برای بارگذاری
تصاویر بافت‌ها در کنار کتابخانه‌ی 
\lr{Assimp}
استفاده شده است.
\cite{stb}


\section{پیاده‌سازی سیستم پویانمایی}

این بخش دو هدف کلی را دنبال می‌کند.

\begin{enumerate}
	\item[-] نمایش مدل گرافیکی و نورپردازی محیط
	\item[-] اجرا و ترکیب کلیپ‌های پویانمایی توسط ماشین حالت متناهی

\end{enumerate}

\subsection{نمایش مدل گرافیکی}

همانطور که گفته شد مدل‌ها یا اشیاء سه‌بعدی به خودی خود مفهومی در 
\lr{OpenGL}
ندارند. آنچه برای 
\lr{OpenGL}
اهمیت دارد لیستی از مثلث‌ها است تا آن‌ها را به تصویر بکشد.
مدل‌های سه‌بعدی از رئوس، لبه و وجوه تشکیل می‌شوند و در فرمت‌های مختلفی مانند
\lr{FBX}
ذخیره می‌شوند. در این پیا‌ده‌سازی، از کتابخانه‌ی 
\lr{Assimp}
برای خواندن این داده‌ها استفاده شده است.

\subsection{قرارگیری مدل سه‌بعدی در کارت گرافیک}

آنچه برای 
\lr{OpenGL}
 اهمیت دارد این است که به آن مجموعه‌ای از مثلث‌ها داده شود تا برایمان ترسیم کند.
برای اینکار به صورت عمومی از 3 آرایه مختلف استفاده می‌شود که به نام‌های 
\lr{VBO}
،
\lr{VAO}
و 
\lr{EBO}
شناخته می‌شوند.
\lr{VBOs}
\LTRfootnote{Vertex Buffer Objectss}
یک آرایه یا بافری است که تمامی رئوس مدل سه‌بعدی ما را در خود جای می‌دهد.

همانطور که در بخش
\ref{Vertex}
اشاره شد، رئوس علاوه بر اینکه شامل اطلاعات موقعیت مکانی در محیط سه‌بعدی هستند، شامل اطلاعات دیگری 
نظیر رنگ، بردار نرمال، مختصات بافت نیز می‌توانند باشند. بنابراین باید به صورتی به کارت گرافیک 
اعلام کنیم که این داده‌ای که در آرایه‌ی 
\lr{VBOs}
قرار دارد را چگونه تفسیر کند.
اینکار با استفاده از یک آرایه‌ی دیگر به نام 
\lr{VAO}
\LTRfootnote{Vertex Array Objects}
صورت می‌گیرد.
در نهایت گفتیم که آنچه برای کارت گرافیک اهمیت دارد دریافت مثلث‌ها است. بنابراین باید به طریقی بگوییم کدارم رئوس با 
اتصال به یکدیگر مثلث تشکیل می‌دهند. اینکار نیز با استفاده از آرایه‌ی 
\lr{EBOs}
\LTRfootnote{Element Buffer Objects}
صورت می‌گیرد.

\subsection{سایه‌زنی فونگ}
پس از نمایش مدل سه‌‌بعدی در محیط گرافیکی به 
بررسی نحوه‌ی نورپردازی آن با استفاده از سایه‌‌زنی فونگ، می‌پردازیم.
همانطور که در 
\ref{PhongShading}
مطرح شد، الگوریتم سایه‌زنی فونگ شامل سه مولفه‌ی اصلی 
نور محیطی
\LTRfootnote{Ambient}
، نور پخش‌شده
\LTRfootnote{Diffuse}
و نور آینه‌وار
\LTRfootnote{Specular}
می‌شود.
در نور محیطی تنها عامل تاثیرگذاری، میزان قدرت منبع نور است.
در نور پراکنده، جهت قرار گیری منبع نور برای ما اهمیت پیدا می‌کند و در نهایت 
در نورپردازی آینه‌وار، موقعیت بیننده به معادله اضافه می‌شود.
هر کدام از این مولفه‌ها یک میزان روشنایی به ما داده و 
رنگ نهایی شئ به صورت زیر محاسبه می‌شود.

\[ result = (ambient + diffuse + specular) * objectColor \]

\subsubsection{نور محیطی}
نور محیطی قدرت نورپردازی منبع نور را برای ما شبیه‌سازی می‌کند.
برای افزودن نور محیطی به اشیاء موجود در صحنه‌ی سه‌بعدی بدین صورت عمل 
می‌کنیم، رنگ نور منبع نور را گرفته و آن را در یک ضریب محیطی 
ثابت کوچک ضرب کرده. این خروجی نورمحیطی
است.

\begin{gather*}
	ambientStrength = 0.1 \\
	ambient = ambientStrength * lightColor
\end{gather*}


\subsubsection{نور پخش‌شده}

هرچه قطعات یک چندضلعی در جهت پرتو‌های نور منبع نور باشند، نور پخش شده 
به آن قسمت روشنایی بیشتری می‌بخشد.
بنابراین این مولفه تاثیر زاویه قرار گیری قطعات شئ با منبع نور را شبیه‌سازی می‌کند.

هر یک از قسمت‌های شی شامل یک بردار نرمال است. بردار نرمال برداری واحد و عمود بر شی است.
با دانستن موقعیت مکانی منبع نور، می‌‌توان برداری از قطعه‌ی موجود در شی 
به منبع نور را بدست آورد.
زاویه‌ی بین این بردار و بردار نرمال، زاویه‌ی بین قطعه‌ی شئ و منبع نور است.
برای اینکه بفهمیم میزان تاثیرگذاری این نور بر روی آن قطعه چقدر است می‌توان
از ضرب نقطه‌ای 
\LTRfootnote{Dot product}
این دو بردار استفاده کرد.
سپس مقدار خروجی را در مقدار رنگ منبع نور کرده و این میزان نور پخش‌شده‌ی 
ما می‌شود.


\begin{gather*}
	norm = normalize(Normal) \\
	lightDir = normalize(lightPos - FragPos) \\
	diff = max(dot(norm, lightDir), 0.0) \\
	diffuse = diff * lightColor
\end{gather*}



\begin{figure}[ht]
	\centerline{\includegraphics[width=\textwidth,height=\textheight,keepaspectratio]{Figures/Ch5/DiffuseLight.png}}

	\caption{نور پخش‌شده}
	\label{fig:DiffuseLight}
\end{figure}
  

\subsubsection{نور آینه‌وار}

مانند نور پخش‌شده، نور آینه‌وار نیز بستگی به زاویه‌ی 
قرار گیری منبع نور و بردار نرمال قطعه دارد. تفاوت این 
مولفه این است که علاوه بر مورد اشاره شده وابسته به جهت مشاهده نیز دارد.

نورپردازی آینه‌وار بستگی بسیاری به خواص بازتابی سطوح دارد.
اگر سطح جسم را مانند یک آینه درنظر بگیریم، جایی را که 
بتوانیم نور منعکس شده را بر روی سطح ببینیم 
شدت آن نور قوی تر خواهد بود.
بنابراین نتیجه‌ی نهایی این نورپردازی بدین صورت است که اجسام از زوایای خاصی 
روشن‌تر به نظر می‌رسند و هرچه از اجسام دورتر شویم از این روشنایی کاسته می‌شود.

برای بدست آوردن شدت این نور نیاز داریم ابتدا زاویه‌ی قرار گیری منبع نور با قسمت مورد 
نظر روی شئ را پیدا کنیم. این زاویه را در نور پخش‌شده نیز بدست آوردیم. 
سپس لازم است آن را نسبت به بردار نرمال شئ بازتاب دهیم.
همچنین لازم است برداری بین بیننده و آن قطعه را نیز بدست آوریم.
در نهایت زاویه‌ی بین بردار نور بازتاب داده‌شده و بردار موقعیت بیننده،
زاویه‌ی مورد نظر ما است و برای بدست آوردن 
تاثیر شدت نور آینه‌وار می‌توان از ضرب نقطه‌ای این دو بردار استفاده کرد.
از آنجایی که این نور بستگی به میزان براق بودن شئ دارد، 
بنابراین مانند فرمول زیر لازم است خروجی ضرب نقطه‌ای را به توان 
یک عددی که این عدد میزان براق بودن است برسانیم.


\begin{figure}[ht]
	\centerline{\includegraphics[width=\textwidth,height=\textheight,keepaspectratio]{Figures/Ch5/SpecularLight_1.png}}

	\caption{نور آینه‌وار}
	\label{fig:SpecularLight_1}
\end{figure}
  

\begin{gather*}
	specularStrength = 0.5 \\
	viewDir = normalize(viewPos - FragPos) \\
	reflectDir = reflect(-lightDir, norm) \\
	spec = pow(max(dot(viewDir, reflectDir), 0.0), 32) \\
	specular = specularStrength * spec * lightColor
\end{gather*}




\begin{figure}[ht]
	\centerline{\includegraphics[width=\textwidth,height=\textheight,keepaspectratio]{Figures/Ch5/SpecularLight_2.png}}

	\caption{بدست آوردن بردار پرتوی منعکس‌شده}
	\label{fig:SpecularLight_2}
\end{figure}


\subsection{اسکلت شخصیت}

اسکلت یک شخصیت به صورت مجموعه‌ای از مفاصل که به صورت سلسله مراتبی به یکدگیر متصل‌اند، تعریف می‌شود.
در این پیاده‌سازی کلاس 
\lr{Bone}
نشان‌دهنده‌ی هر مفصل است.
هر 
\lr{Bone}
یک والد دارد و می‌تواند به هر تعدادی فرزند داشته باشد.
با توجه به تعریف آورده شده از اسکلت، کلاس اسکلت که با
\lr{Skeleton}
مشخص شده، شامل لیستی از این مفاصل به همراه اشاره‌گری به مفصل ریشه است.

\subsection{اتصال اسکلت و مدل سه‌بعدی}
اصطلاحی که برای اتصال اسکلت و مدل سه‌بعدی استفاده می‌شود
\lr{Skinning}
است.
در این روش هر راس موجود در مدل، به یک یا چند مفصل متصل می‌شود.
الگوریتم به کار‌رفته در این پیاده‌سازی،الگوریتم
\lr{linear blend skinning}
نام دارد. در این الگوریتم زمانی که یک راس به یک مفصل می‌شود به آن یک وزن نسبت داده می‌شود.
این وزن نشان‌دهنده‌ی میزان تاثیرگذاری این مفصل بر روی این راس است.
به بیانی دیگر، این وزن نشان می‌دهد که اکر این مفصل به مکان جدید منتقل شود، این انتقال چقدر بر روی آن راس تاثیر می‌گذارد.
بنابراین برای بدست آوردن انتقال نهایی راس، باید انتقال راس را نسبت به هرکدام از مفاصلی که به آن متصل است را بدست آوریم، سپس انتقال نهایی
برابر مجموع وزن‌دار تمامی این انتقال‌ها خواهد بود. 

فرمول نهایی انتقال هر راس بر اساس این الگوریتم به صورت زیر محاسبه می‌شود.

\[ v_M^C = \sum_{i=1}^{N} w_i^j K_i v_M^B \]


مقدار  \( v_M^C \)
بیانگر مکان راس نسبت به مختصات ریشه مش در حالت فعلی است.

مقدار  \( v_M^B \)
بیانگر مکان راس نسبت به مختصات ریشه مش در هنگامی که مش در حالت اتصال قرار دارد، است.
ماتریس \( K_i \)
ماتریس انتقال است که بر روی راس انجام می‌شود.
مقدار \( w_i^j \)
بیانگر میزان وزن تاثیرگذاری مفصل \(j\) 
بر روی این راس است.
مقدار \( N \)
بیانگر تعداد مفاصلی است که روی این راس تاثیر می‌گذارند.


برای انتقال رئوس از ماتریس انتقال \(K_i\) 
استفاده شده است.
این ماتریس ابتدا راس را از فضای مختصاتی مش به فضای مفصل مورد نظر برده، سپس مفصل را با استفاده از ماتریس انتقال کلیپ پویانمایی،
انتقال داده و پس از آن راس را دوباره به فضای مختصاتی مش باز‌ می‌گرداند.
زمانی که رئوس در فضای مفصل قرار می‌گیرند و پس از آن مفصل 
انتقال پیدا می‌کند، مکان راس نسبت به مفصل تغییر نمی‌کند.
بنابراین به همین علت است که ابتدا رئوس به فضای مفصل موردنظر برده می‌شود و سپس آن مفصل را انتقال داده و در نهایت 
راس را به فضای مختصاتی مش باز‌می‌گرداند.

می‌توان حالت باز‌شده‌ی این ماتریس را در فرمول زیر مشاهده کرد.


\[ K_i =  P_{j \rightarrow M} A_J B_{M \rightarrow J} \]

این ماتریس‌ها برای انتقال و تغییر فضای مختصات استفاده می‌شوند.
زمانی که ماتریس \(K_i\)
در راس ضرب می‌شود، اولین ماتریسی که روی راس تاثیر می‌گذارد، ماتریس 
\(B_{M \rightarrow J}\)
است. این ماتریس فضای مختصات راس را از فضای مختصات مش به فضای مختصات آن مفصل مورد نظر تغییر می‌دهد.
پس از این ماتریس، ماتریس انتقال 
\(A_J\) 
اعمال می‌شود که این ماتریس از کلیپ پویانمایی برای این مفصل بدست می‌آید.
و درنهایت ماتریس 
\(P_{j \rightarrow M}\)
اعمال می‌شود که این ماتریس، راس را به فضای مختصاتی مش باز می‌گرداند.



\subsection{کلیپ پویانمایی}
در بازی‌های کامپیوتری هر کلیپ پویانمایی شامل یک حرکت منحصر به فرد شخصیت داخل بازی است.
هر کلیپ‌ شامل ژست‌های اسکلت در فاصله‌های زمانی مشخصی است. در واقع آنچه باعث حرکت شخصیت می‌شود حرکت اسکلت شخصیت است.
زمانی که اسکلت شخصیت با استفاده از یک کلیپ جابه‌جا می‌شود، مدل شخصیت نیز با استفاده از روش‌های 
\lr{skinning}
که در بالا توضیح داده‌شد همراه این اسکلت حرکت می‌کند.

کلیپ‌های پویانمایی از طریق کلاسی به اسم
\lr{Animation Clip}
مدل‌سازی شده اند. 
این کلاس شامل آرایه‌ای از ژست‌های شخصیت در مدت زمان‌های مشخصی است. همراه یک اشاره‌گری به اسکلت شخصیت.
نکته‌ی قابل توجه این است که هر کلیپ پویانمایی مربوط به یک نوع اسکلت می‌شود. به زبانی دیگر نمی‌توان کلیپی که برای اسکلت
شخصیت انسانی طراحی شده است را بر روی یک حیوان، مانند فیل اجرا کرد.

\begin{figure}[ht]
	\centerline{\includegraphics[width=\textwidth,height=\textheight,keepaspectratio]{Figures/Ch5/AnimationClip.png}}

	\caption{نمایی از ژست شخصیت در یک کلیپ پویانمایی}
	\label{fig:AnimationClip}
\end{figure}


\subsection{پخش‌کننده‌ی کلیپ‌های پویانمایی}

این سیستم وظیفه‌اش پخش کردن کلیپ پویانمایی بر روی اسکلت شخصیت است.
این سیستم با گرفتن یک کلیپ و یک اسکلت، این کلیپ را بر روی آن اسکلت اجرا می‌کند.
همانطور که گفتیم، کلیپ‌های پویانمایی ژست شخصیت را در فاصله‌های زمانی مشخصی در خود ذخیره ‌می‌کنند. وظیفه‌ی این سیستم این است که با استفاده از یک زمان‌سنج که نشان‌دهنده‌ی زمان فعلی بازی است ژست مناسب شخصیت را از داخل کلیپ بدست آورد.
قابل ذکر است که ممکن است این ژست با توجه به زمان بازی و فاصله‌های زمانی داخل کلیپ
از درون‌یابی دو ژست پشت سر هم در آن کلیپ بدست آید.

\subsection{الگوریتم پخش‌کننده‌ی کلیپ‌های پویانمایی }

هر شخصیت درون بازی، اگر از نوع شخصیت اسکلتونی باشد، دارای یک پخش کننده‌ی کلیپ پویانمایی خواهد بود.

در تصویر زیر تابع به‌روزرسانی اسکلت به وسیله‌ی کلیپ پویانمایی را می‌توان مشاهده کرد.

\begin{latin}
	\begin{lstlisting}
		currentTime += deltaTime;

		const double currentAnimationTime = (currentTime - startTimeForCurrentAnim);
		AnimationPose currentPose = currentClip->GetPoseForCurrentFrame(currentAnimationTime * currentClip->GetFramePerSecond());
		
		SetSkeletonPose(currentPose);
	\end{lstlisting}
\end{latin}	



برای اینکه بتوان یک کلیپ را پخش کرد نیاز است دو مورد زیر را بدانیم.

\begin{enumerate}
	\item[-] زمان فعلی درون بازی(\lr{CurrentTime})
	\item[-] زمان شروع پخش کلیپ فعلی(\lr{StartTimeForCurrentAnimation}) 
\end{enumerate}

در ابتدا زمان فعلی درون بازی را برای این پخش‌کننده به‌روزرسانی می‌کنیم.
سپس برای بدست آوردن زمان فعلی کلیپ می‌توان از فرمول زیر استفاده کرد

\[CurrentAnimationTime = CurrentTime - StartTimeForCurrentAnimation\]

در نهایت با استفاده از این مقدار می‌توان ژست مورد نظر را از داخل کلیپ پویانمایی بدست ‌آورد.
در نهایت نیز این ژست را بر روی اسکلت شخصیت اعمال می‌کنیم.


\subsection{ماشین حالت پویانمایی}

یکی از روش‌های ترکیب کلیپ‌های مختلف با یکدیگر، استفاده از ماشین حالت متناهی است.
یک ماشین‌ حالت متناهی شامل چندی حالت مختلف است
که هر کدام از این حالات، حالتی از وضعیت سیستم را مشخص می‌کنند.
زمانی که از ماشین حالت استفاده می شود سیستم می‌تواند در هر لحظه تنها در یکی از این حالات قرار گیرد.
البته سیستم می‌تواند با دریافت ورودی از یک حالت به حالت دیگری رود.

دلیل استفاده از ماشین حالت متناهی برا سیستم پویانمایی این است که همانگونه که گفتیم، کلیپ‌های پویانمایی، شامل ویدیو‌های کوتاهی هستند که یک حالت مشخصی از شخصیت را بیان می‌کنند.
در یک بازی، با توجه به ورودی بازیکن، شخصیت درون بازی می‌تواند در حالت‌های متفاوتی قرار گیرد. با استفاده از ماشین حالت می‌توان به تمامی این حالت‌ها رسیدگی کرد.

به عنوان مثال، تصویر 
\ref{fig:LocomotionStateMachine}
نشان‌دهنده‌ی یک ماشین‌حالت برای حرکت شخصیت است. شخصیت در ابتدا در حالت ایستاده قرار دارد و با گرفتن
ورودی‌های مختلف از کیبورد، می‌تواند به حالت‌های دیگری رود.

\begin{figure}[ht]
	\centerline{\includegraphics[width=\textwidth,height=\textheight,keepaspectratio]{Figures/Ch5/LocomotionStateMachine.png}}

	\caption{ماشین حالت برای حرکت شخصیت}
	\label{fig:LocomotionStateMachine}
\end{figure}


برای پیا‌ده‌سازی ماشین حالت متناهی، این سیستم به سه کلاس کلی شکسته‌شده‌است. کلاس
\lr{AnimationStateMachine}
که وظیفه‌ی مدیریت حالت‌ها و انتقال از یک حالت به حالت دیگری را دارد.
کلاس 
\lr{AnimationState}
که نشان‌دهنده‌ی حالت شخصیت است. هر 
\lr{AnimationState}
شامل یک کلیپ است و هر زمانی که این حالت فعال می‌شود این کلیپ پخش می‌شود.
 در نهایت کلاس
\lr{Transition}
که شامل توابع انتقال است.
هر حالت می‌تواند شامل چندین انتقال باشد. و وظیفه‌ی
\lr{AnimationStateMachine}
است که بررسی کند، اگر انتقالی امکان‌پذیر بود، آن را انجام دهد.

\subsection{به‌روزرسانی ماشین حالت پویانمایی}

وضعیت توابع انتقال تاثیرگذاری مستقیمی در وضعیت سیستم به‌روزرسانی ماشین حالت دارد.
وضعیت انتقال می‌تواند سه حالت زیر را داشته باشد.

\begin{enumerate}
	\item[-] حالت عادی \LTRfootnote{Normal}
	\item[-] حالت در حال انتقال \LTRfootnote{Transitioning}
	\item[-] حالت اتمام انتقال \LTRfootnote{Finished}
\end{enumerate}


حالت اول حالت عادی
است که نشان‌دهنده‌ی وضعیت عادی ماشین حالت است. در این وضعیت، توابع انتقال حالت فعلی بررسی می‌شوند تا در صورتی که شرایطشان برقرار شود، تغییر حالت رخ دهد. علاوه بر آن کلیپ حالت فعلی با استفاده از کلاس پخش‌کننده آپدیت می‌شود.

در صورتی که توابع انتقال مقدار درست
\LTRfootnote{True}
را بازگردانند، ماشین به وضعیت دوم که وضعیت درحال انتقال
است، تغییر وضعیت می‌دهد.
در این وضعیت با توجه به زمانی که مشخص شده، ژست شخصیت با استفاده از درون‌یابی خطی از حالت فعلی به حالت جدید تغییر می‌کند.

پس از اینکه انتقال به صورت کامل انجام شد، وضعیت ماشین حالت به اتمام انتقال
تغییر می‌یابد. زمانی که ماشین‌ در این وضعیت قرار گرفته یعنی به حالت جدید منتقل شده، بنابراین لازم است کلیپ پویانمایی را از حالت جدید گرفته و آن را به کلاس پخش کننده داده تا آن را پخش کند.
پس از این کار وضعیت ماشین دوباره به حالت عادی تغییر می‌یابد و همه‌ی این موارد دوباره تکرار می‌شوند.

\begin{latin}
	

\begin{lstlisting}

	if(transitionStatus == TransitionStatus::normal) 
	{
		for (const auto&  transition : currentState->GetTransitions()) // loop through transitions of the current state
		{
			if (transition->Evaluate())
			{
				transitionStatus = TransitionStatus::transitioning;
				currentState = animationStatesMap.at(transition->to);
				TransitionFromPose = animator->GetPoseAtCurrentTime();
				TransitionToPose = currentState->GetAnimClip()->GetPoseForCurrentFrame(0);
				currentTime = 0;
				transitionTime = transition->transitionTime;
				break;
			}
		}
	}

	if (transitionStatus == TransitionStatus::normal)
	{
		animator->Update(deltaTime);
	}
	else if(transitionStatus == TransitionStatus::transitioning)
	{
		if(TransitionUpdate(deltaTime)) 
		{
			transitionStatus = TransitionStatus::finished;
		}
	}
	else if(transitionStatus == TransitionStatus::finished)
	{
		animator->ChangeAnimationClip(*(currentState->GetAnimClip()), 0); 
		transitionStatus = TransitionStatus::normal;
	}
\end{lstlisting}

\end{latin}




\section{نتیجه‌گیری}

خروجی نهایی این فصل، یک برنامه‌‌ی گرافیکی برای بارگذاری اشیاء سه‌بعدی است.
از آنجایی که این اشیاء می‌‌توانند شامل مش‌های اسکلتی باشند، بنابراین توانایی اجرای 
کلیپ‌های پویانمایی را خواهند داشت. در نهایت با پیاده‌سازی سیستم ترکیب با استفاده از 
ماشین حالت متناهی، این مش‌های اسکلتی می‌توانند در حالت های مختلف قرار گیرند 
و بر اساس هر کدام از این حالات، کلیپ متفاوتی را پخش کنند.
