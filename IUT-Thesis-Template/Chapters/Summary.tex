\chapter {نتیجه‌گیری }

هدف این پروژه آشنایی با محیط‌های گرافیکی و الگوریتم‌های موجود در آن با 
تاکید بر سیستم پویانمایی کامپیوتری بود.
برای بدست آوردن این هدف، در این پروژه ابتدا به بررسی سیستم گراف 
پویانمایی یکی از بزرگترین موتور‌های بازی‌سازی جهان، یعنی موتور بازی‌سازی آنریل 
پرداختیم.
با این بررسی متوجه‌‌شدیم که یک سیستم پویانمایی چه ابزارهایی را در اختیار کاربران 
قرار می‌دهد و نحوه‌ی کلی استفاده از این ابزار‌ها چگونه است.

در نهایت برای تحلیل عمیق این سیستم‌ها به پیاده‌سازی یک سیستم مشابه 
با استفاده از واسط برنامه نویسی کاربردی 
\lr{OpenGL}
پرداختیم.
خروجی این پیاده‌سازی، یک نرم‌افزاری گرافیکی است که به وسیله‌ی آن 
می‌توان مش‌های اسکلتی را بارگذاری کرد و روی آن‌ها کلیپ‌های 
پویانمایی مختلفی را اجرا کرد.
