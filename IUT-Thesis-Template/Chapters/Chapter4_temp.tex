\chapter {سیستم انیمیشن گراف در موتور بازی‌سازی آنریل}

در این فصل ابتدا توضیحاتی راجع به آنریل انجین داده می‌شود و سپس در رابطه‌ی سیستم انیمشن گراف این انجین صحبت خواهد شد.

\section{بازیگران، پیاده‌ها و شخصیت‌‌ها}

اشیا در آنریل می‌توانند به سه کلاس کلی بازیگران، پیاده‌ها و شخصیت‌ها دسته‌بندی می‌شوند.
بازیگران کلاس پایه‌ی تمامی اشیا ای هستند که به صورت فیزیکی می‌توانند در محیط سه‌بعدی قرار گیرند.
پیاده‌ها کلاسی مشتق شده از بازیگران هستند که بازیکنان می‌توانند کنترل آن‌ها را بدست گیرند و 
در محیط حرکت کنند.
 و در نهایت شخصیت‌ها پیاده‌هایی هستند که دارای مش اسکلتونی، توانایی شناسایی برخورد و منطق حرکتی هستند.
 آنها مسئول تمام تعاملات فیزیکی بین بازیکن یا هوش مصنوعی، با جهان هستند و همچنین مدل های اولیه شبکه و دریافت ورودی را پیاده سازی می کنند. 
اگر بخواهیم شخصیت درون بازی از انیمیشن‌های اسکلتونی استفاده کند، باید از این کلاس بهره ببریم.

\section{اجزاء}

اجزاء
\LTRfootnote{Components}
مجموعه‌ای از توابع و ویژگی‌ها است که می‌تواند به یک بازیگر اضافه شود.
بنابراین بازیگران می‌توانند حاوی مجموعه‌ای از
\lr{ActorComponents}
باشند که این اجزاء می‌توانند برای موارد مختلفی از جمله
کنترل نحوه‌ی حرکت بازیگران، 
نحوه‌ی رندر شدن و غیره استفاده شوند.

زمانی که یک مولفه به یک بازیگر اضافه می‌شود، آن بازیگر می‌تواند عملکرد‌های موجود در آن مولفه را استفاده کند.
به عنوان مثل یک مولفه نور نقطه‌ای باعث می‌شود که بازیگر مانند یک نور نقطه‌ای، نور ساطع کند.
یا یک مولفه صورتی به بازیگر این توانایی پخش صدا را می‌دهد.

مولفه‌ها حتما باید به یک بازیگر متصل شوند و به خودی خود نمی‌توانند وجود داشته باشند.
درواقع وقتی ما مولفه‌های مختلف را به بازیگر خود متصل می‌کنیم درواقع در حال قرار دادن قطعه‌ها و تکه‌هایی هستیم
 که مجموع آن‌ها یک بازیگر را به عنوان یک موجودیت واحد که در محیط سه‌بعدی قرار می‌گیرد تعریف می‌کنند.
 به عنوان مثال چرخ‌های یک ماشین، فرمان ماشین، چراغ‌ها و غیره همه به عنوان
 مولفه‌های ماشین درنظر گرفته می‌شوند در حالی که خود آن ماشین، بازیگر است.

\section{شخصیت‌ها}

هر شخصیت در آنریل از سه مولفه‌ی اصلی تشکیل شده است.


\begin{itemize}
	\item \lr{Skeletal Mesh Component}
	\item \lr{Character Movement Component}
	\item \lr{Capsule Component}
\end{itemize}


همانطور که در فصل‌های گذشته اشاره‌شد شخصیت‌ها برای پخش انیمیشن‌‌ها نیاز به یک مش اسکلتونی دارند.
مولفه‌ی 
\lr{Skeletal mesh Component }
 مش اسکلتونی اصلی مرتبط با شخصیت است.
این مولفه‌ای است که برای ما در این پروژه اهمیت زیادی دارد.

مولفه‌ی 
\lr{ Character Movement Component}
همانطور که از اسمش مشخص است برای منطق حرکت در حالت‌های مختلف از جمله راه‌رفتن افتادن و غیره استفاده می‌شود.
این مولفه شامل تنظیمات و عملکرد‌های مربوطه برای کنترل حرکت است.

و در نهایت مولفه‌‌ی
\lr{Capsule Component}
وظیفه‌ی تشخیص برخورد در هنگام حرکت را دارد.


