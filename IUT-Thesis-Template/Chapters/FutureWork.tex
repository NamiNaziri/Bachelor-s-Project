\chapter {کار‌‌های آینده}

نرم‌افزار گرافیکی پیاده‌سازی شده می‌تواند از جهات مختلفی گسترش یابد.
 یکی از مواردی که می‌توان اشاره کرد 
ایجاد ویژگی‌های جدید به نرم‌افزار فعلی است. همانطور که در تحقیق درباره‌ی سیستم آنریل متوجه‌ شدیم، سیستم‌‌های پویانمایی،
سیستم‌‌های بسیار گسترده‌ای هستند. ویژگی‌هایی مانند، اضافه کردن الگوریتم‌های مختلف برای ترکیب،
اضافه کردن پویانمایی بر اساس فیزیک، اضافه کردن 
مواردی همچون سینماتیک معکوس برای ایجاد پویانمایی رویه‌ای می‌توانند تنها سطحی از دریای عمیق ویژگی‌ها باشند.

علاوه بر این، از آنجایی که این برنامه‌ی در حال حاضر از کتابخانه‌ی 
\lr{Assimp}
برای بارگذاری مدل‌های سه‌بعدی استفاده می‌کند، سرعت مناسبی ندارد.
می‌توان با نوشتن یک سیستم جداگانه برای بارگذاری اشیاء به سرعت این بارگذاری افزود.

علاوه بر این موارد، این برنامه را می‌توان از جهت موتور بازی‌سازی نیز ارتقا بخشید.
به عنوان مثال، در سیستم فعلی هیچگونه الگوریتم برخوردی، پیاده‌سازی نشده است. با پیاده‌سازی چنین مواردی،
می‌توان به واقع‌گرایانه‌تر شدن این برنامه کمک کرد.