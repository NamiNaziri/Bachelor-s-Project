
\section{موتور بازی‌سازی}
موتور‌های بازی‌سازی پلتفرم‌هایی هستند که ساخت بازی‌های رایانه‌ای را آسان‌تر می‌کنند.
آن ها به شما این امکان را می‌دهند تا عناصر بازی مانند پویانمایی، تعامل با کاربر یا تشخیص برخورد میان اشیاء را در یک واحد ادغام و ترکیب کنید.
\cite{barczak2019comparative}
زمانی که از اصطلاح موتور بازی‌سازی استفاده می‌کنیم منظورمان نرم‌افزارهای قابل توسعه‌ای هستند که می توانند پایه و اساس بسیاری از بازی‌های مختلف باشند.
\cite{GameEngineArchitecture}
موتورهای بازی‌سازی متشکل از اجزای مختلفی هستند که قابلیت‌های لازم برای ساخت بازی را فراهم می‌کنند.
رایج ترین اجزای موتور بازی عبارتند از:
\cite{barczak2019comparative}
\begin{itemize}
    \item[-] مولفه‌ی صدا: نقش اصلی این مولفه تولید جلوه‌های صوتی در بازی است.
    \item[-] موتور رندر: وظیفه اصلی این مولفه تبدیل داده‌های ورودی به پیکسل‌ها، برای به تصویر کشاندن بر روی صفحه است.
    \item[-] مولفه هوش مصنوعی: این مولفه مسئولیت ارائه‌ی تکنیک‌هایی برای تعریف قوانین رفتار شخصیت‌هایی را دارد که توسط بازیکنان کنترل نمی‌شوند.
    \item[-] مولفه پویانمایی: نقش اصلی این مولفه اجرای کلیپ‌های پویانمایی مختلف مانند حرکت است.
    \item[-] مولفه شبکه: وظیفه اصلی این مولفه قادرساختنِ بازیِ همزمانِ بازیکنان با یکدیگر، از طریق استفاده از دستگاه‌های متصل به اینترنت است.
    \item[-] مولفه منطق یا مکانیک بازی: این مولفه قوانین حاکم بر دنیای مجازی، ویژگی‌های شخصیت‌های بازیکنان، هوش مصنوعی و اشیاء موجود در دنیای مجازی و همچنین وظایف و اهداف بازیکنان را تعریف می‌کند.
    \item[-] ابزارهای نرم‌افزاری: وظیفه اصلی این ابزارها افزایش راندمان و سرعت تولید بازی با موتور بازی‌سازی است. آن‌ها توانایی اضافه‌کردن بسیاری از عناصر مختلف را به بازی‌ها، از پویانمایی و جلوه‌های صوتی گرفته تا الگوریتم‌های هوش مصنوعی، را فراهم می‌کنند.   
\end{itemize}

یکی از مهم‌ترین مولفه‌های موجود در هر موتور بازی، مولفه‌ی پویانمایی آن است. در این پروژه به بررسی سیستم
 گراف پویانمایی که وظیفه‌ی پخش کلیپ‌های پویانمایی شخصیت‌های سه‌بعدی را در موتور بازی آنریل دارد می‌پردازیم.



\section {موتور بازی‌سازی آنریل}

اولین نسل موتور بازی‌سازی آنریل توسط تیم سوینی، بنیانگذار اپیک گیمز
\LTRfootnote {Epic Games}
،
توسعه یافت.
سویینی در سال 1995 شروع به نوشتن این موتور برای تولید بازی‌ تیراندازی اول شخصی به اسم غیرواقعی
\LTRfootnote{Unreal}
کرد.


نسخه‌‌ی دوم موتور بازی‌سازی آنریل در سال 2002 منتشر شد. 

نسخه سوم نیز در سال 2004 پس از حدود 18 ماه توسعه، منتشر شد.
در این نسخه، معماری پایه‌ای موجود در نسخه‌ی اول مانند طراحی شی‌گرا، اسکریپت‌نویسی مبتنی بر داده و رویکرد نسبتا ماژولار نسبت به زیرسیستم‌ها وجود داشت.
اما برخلاف نسخه دوم که از یک خط لوله با عملکرد ثابت
\LTRfootnote{fixed-function pipeline}
استفاده می‌کرد، این نسخه به صورتی طراحی شده بود تا بتوان قسمت‌های سایه‌زنی سخت‌افزاری
\LTRfootnote{shader hardware}
را برنامه‌نویسی کرد.


موتور بازی‌سازی آنریل 4 در سال 2014 در کنفرانس توسعه‌دهندگان بازی
\LTRfootnote{GDC}
منتشر شد.
این نسخه با طرح کسب‌و‌کار اشتراکی برای توسعه‌دهندگان در دسترس قرار گرفت. این اشتراک به صورت ماهانه، با پرداخت 19 دلار آمریکا به توسعه‌دهندگان این اجازه را می‌داد تا به نسخه‌ی کامل موتور، از جمله کد منبع 
\lr {C++}
آن
دسترسی پیدا‌ کنند.
البته در سال 2015 اپیک گیمز موتور بازی‌سازی آنریل را به صورت رایگان برای همگان منتشر ساخت.

آخرین نسخه آنریل به اسم موتور بازی‌سازی آنریل 5 در سال 2020 معرفی شد. این نسخه از تمام سیستم‌های موجود از جمله کنسول‌های نسل بعدی پلی‌استیشن 5
\LTRfootnote{PlayStation 5}
و ایکس‌باکس سری 
\lr{X/S}
\LTRfootnote{Xbox Series X/S}
پشتیبانی می کند.
کار بر روی این موتور حدود دو سال قبل از معرفی آن شروع شده بود. در سال 2021 نسخه‌ای از آن به صورت دسترسی اولیه منتشر شد. به طور رسمی در سال 2022 نسخه‌ی کامل این موتور برای توسعه‌دهندگان انتشار یافت.
\cite{UnrealEngineWikiPedia}

\section{زبان برنامه‌نویسی در آنریل}

موتور بازی‌سازی آنریل از زبان برنامه نویسی 
\lr{C++}
به همراه اسکریپ بصری به نام 
\lr{Blueprint}
استفاده می‌کند.

\lr{Blueprint}
یک سیستم برنامه‌نویسی کامل گیمپلی مبتنی بر مفهوم استفاده از رابط‌های مبتنی بر گره برای ایجاد عناصر گیمپلی از درون ویرایشگر است.
این سیستم بسیار منعطف و قدرتمند است زیرا این توانایی را در اختیار طراحان قرار می دهد تا از طیف گسترده ای از مفاهیم و ابزارها که عموماً فقط در دسترس برنامه نویسان هستند استفاده کنند.
\cite{UnrealEngineBlueprint}
