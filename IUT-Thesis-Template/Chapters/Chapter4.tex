\chapter { انیمیشن‌های اسکلتونی }

%{
%در این فصل برای پیا‌ده‌سازی سیستم انیمیشن به دنبال ارضا کردن 4 هدف کلی هستیم.
%\begin{enumerate}
%	\item نمایش اسکلت شخصیت در یک محیط سه‌بعدی
%	\item نمایش انیمیشن‌های سه‌بعدی اسکلت
%	\item نمایش شخصیت دارای مدل و اجرای انیمیشن بر روی آن
% 	\item با پیاده‌سازی روش ترکیب برای ترکیب انیمیشن‌های مختلف با یکدیگر
%\end{enumerate}

انیمیشن‌ اسکلتونی تکنیکی در انیمیشن‌های کامپیوتری است که در آن شخصیت درون بازی به دوبخش تقسیم ‌می‌شود. یک بخش،یک مش یا پوسته است که برای به نمایش کشاندن ‌آن شخصیت در محیط سه‌بعدی استفاده می‌شود و بخش دوم یک اسکلت است. این اسکلت مجموعه سلسله مراتبی از قطعات به‌ هم پیوسته است که به هر قطعه یک مفصل گویند.
در این فصل به بررسی این دوبخش و تکنیک‌های موجود در انیمیشن‌های اسکلتونی خواهیم پرداخت

\section{مدل اسکلتونی}

در انیمیشن‌های اسکلتونی از مدل‌های اسکلتونی استفاده می‌شود. هر مدل اسکلتونی از دو بخش مدل و اسکلت تشکیل شده است. 

\section{شبکه‌ی\protect\LTRfootnote{Mesh} چندضلعی}

.در گرافیک کامپیوتری سه‌بعدی و مدل‌سازی جامد، شبکه چند‌ضلعی مجموعه‌ای از رئوس، لبه‌ها و وجوه است که شکل یک جسم چند‌وجهی را مشخص می‌کند
وجه‌ها معمولاً از مثلث‌ها (شبکه مثلثی)، چهار ضلعی (چهار گوشه)، یا دیگر چند ضلعی‌های محدب ساده
(
	\lr{n}
	ضلعی‌ها
)
تشکیل شده‌اند. دلیل استفاده از این نوع چند ضلعی‌ها آسان‌تر بودن به نمایش کشیدن وجوه در محیط سه‌بعدی است.
البته در حالت کلی اشیاء ممکن است از چندضلعی‌های مقعر و یا حتی چندضلعی‌های دارای سوراخ نیز تشکیل شده باشند.

اشیاء ایجادشده توسط مش‌های چند ضلعی باید انواع مختلفی از عناصر، از جمله رئوس، لبه‌ها، وجوه، چندضلعی‌ها و سطوح را ذخیره کنند.
در بسیاری از نرم‌افزارهای سه‌بعدی، فقط رئوس، لبه‌ها و یکی از دو مورد وجوه یا چند‌ضلعی‌ها ذخیره می‌شوند.
در اکثر سیستم‌های رندرکننده
\LTRfootnote{renderer}
فقط از وجوه سه‌ضلعی
(مثلث‌ها)
استفاده‌ می‌شود.
بنابراین در این حالت چند‌ضلعی‌های مدل، باید به شکل مثلث باشند. البته سیستم‌های رندر ای وجود دارند که از چهارضلعی‌ها یا چندضلعی‌های با تعداد اضلاع بالاتر نیز پشتیبانی ‌می‌کنند و یا در لحظه این چندضلعی‌ها را به مجموعه‌ای از مثلث‌ها تبدیل می‌کنند که در این صورت باعث ‌می‌شود نیازی به ذخیره‌ی مش به شکل مثلثی نباشد.

بنابراین چهار قسمت اصلی یک مش چندضلعی، رئوس، لبه‌ها، وجوه و چندضلعی‌ها هستند. توضیح کوتاهی درباره‌ی هر کدام از این موارد را در بخش زیر می‌توانیم مشاهده کنیم.

\subsection{راس}
راس‌ها معمولا یک موقعیت در فضای سه‌بعدی همراه با اطلاعات دیگر مانند رنگ، بردار نرمال، مختصات بافت 
در راس‌های مربوط به مش‌های اسکلتونی اطلاعاتی مانند تعداد مفاصلی که بر روی این راس تاثیر می‌گذارد همراه با وزن تاثیرگذاری‌اش می‌تواند اضافه شود.

\subsection{لبه}
ارتباط بین دو راس را لبه گویند.

\subsection{وجه}
مجموعه‌ای بسته از لبه‌ها را وجه گویند. وجه‌ها می‌توانند از سه لبه 
(وجه مثلثی)
یا از چهار لبه
(وجه چهارگوش)
تشکیل شده باشند.

\subsection{چندضلعی}
یک چندضلعی مجموعه‌ای همسطح از وجود است.
در سیستم‌هایی که از وجه‌های چند ضلعی پشتیبانی می‌کنند، وجوه و چندضلعی‌ها یکسان هستند ولی در صورتی که سیستم مورد نظر تنها از سه یا چهار ضلعی‌ها پشتیبانی کند، در این صورت چند ضلعی‌ها را مجموعه‌ای از وجوه گویند.

\section{مدل}

مدل‌
\footnote{ گاهی به جای استفاده از واژه‌ی مدل، از واژه‌ی مش هم استفاده می‌شود.}
درواقع هر شئ‌ای است که در محیط سه‌بعدی قرار می‌گیرد و به تصویر کشیده ‌می‌شود. هر مدل می‌تواند از چند زیرمش تشکیل شود.
به عنوان مثال یک ماشین را درنظر بگیریم. موجودیت ماشین می‌تواند یک مدل باشد که در محیط سه‌بعدی قرار می‌گیرد. مدل ماشین می‌تواند از چند زیرمش مانند چرخ‌ها، لاستیک‌ها و بدنه‌ی ماشین تشکیل شود. دلیل وجود داشتن یک موجودیت کلی به اسم ماشین این است که یک ‌شخصی مانند طراح محیط و یا طراح مرحله ‌نمی‌خواهد هر بار که ماشینی را در محیط قرار دهد، تک تک زیرمش‌ها را به صورت دستی در صحنه وارد کند و در سر جای خودش قرار بدهد.


\section{زیرمش
\protect\LTRfootnote{Sub-Mesh}
}
چندضلعی‌های دارای یک نوع ماده
\LTRfootnote{Material}
را یک زیرمش گویند.
همانطور که اشاره شد، هر مدل از چند زیرمش تشکیل می‌شود. دلیل این تقسیم این است که در هر عملیات به تصویر کشیدن
\LTRfootnote{Render}
تنها یک ماده می‌تواند به تصویر کشیده شود. مثلا در همان مثال ماشین، قسمت‌های مختلف ماشین از ماده‌های مختلفی تشکیل می‌شود. به طور مثال چرخ ماشین می‌تواند از جنس آلومینیوم باشد یا لاستیک چرخ از جنس پلاستیک باشد و یا حتی قسمت‌های داخلی ماشین مانند صندلی ماشین از جنس چرم باشد.
بنابراین باید این قسمت‌ها به صورت جدا قرار گیرند تا بتوان هر قسمت را با توجه به ماده‌ی موردنظر آن به تصویر کشاند.

\section{ماده 
\protect\LTRfootnote{Material}
}
ماده‌ها شامل پارامتر‌های قابل تنظیمی هستند که با تنظیم آن به گرافیک ما اعلام می‌کند که چگونه باید یک مثلث را به تصویر بکشد.
این پارامترها می‌توانند شامل موارد زیر باشند ولی محدود به آن نمی‌شوند

\begin{enumerate}
	\item میزان کدورت و شفافیت شئ
 	\item میزان براقی شئ
 	\item رنگ(بافت) شئ
 	\item سایه‌زنی پیکسلی یا راسی \protect\LTRfootnote{Vertex or Pixel shader}
\end{enumerate}


\section{بافت
\protect\LTRfootnote{Texture}
}
بافت یک تصویر دوبعدی و یا سه‌بعدی است که می‌تواند در ماده استفاده شود.
این تصاویر به عنوان ورودی در برنامه دریافت شده و پس از اینکه یک شناسه به آن ها تخصیص داده شد، در کارت گرافیکی قرار می‌گیرند. ماده‌ها با استفاده از این شناسه می‌توانند در صورت لزوم به این بافت دستیابی پیدا کنند.


\section{اسکلت}

به مجمو‌عه ای از مفاصل که به صورت سلسله مراتبی به یکدیگر متصل می‌شوند، اسکلت گویند. پس از آنکه هنرمندان مدل شخصیت را طراحی می‌کنند در طی یک مرحله که به آن
\lr{Rigging}
گویند، ساختار سلسله مراتبی اسکلت را به وجود می‌آورند.
در انیمیشن‌ها درواقع این اسکلت‌ است که حرکت می‌کند و با حرکتش باعث حرکت مدل شخصیت می‌شود.


\section{
\lr{Skinning}
}

تا اینجا با دو مفهوم مدل و اسکلت آشنایی پیدا کردیم ولی نگفتیم که این دو چگونه به هم مرتبط می‌شوند.
به عملیاتی که طی آن مفاصل موجود در اسکلت به مدل متصل می‌شود 
\lr{skinning}
گویند.
طی این مرحله هر راس موجود در پوسته‌ی مش به یک یا چند مفصل متصل می‌شود.
برای اینکه چگونه رئوس مش، این مفاصل را دنبال کنند الگوریتم‌های مختلفی مطرح شده است که در فصل پیاده‌سازی به آن اشاره ‌خواهد شد.

\section{‌کلیپ‌های انیمیشنی}

در یک فیلم انیمیشنی، تمام بخش‌های یک صحنه قبل از ساخت هر انیمیشن به دقت برنامه‌ریزی می‌شود.
این شامل حرکات هر شخصیت، لوازم موجود در صحنه و حتی حرکات دوربین نیز می‌شود.
این بدان معنی است که کل صحنه را می‌توان به عنوان یک دنباله طولانی و پیوسته از فریم‌ها، متحرک ساخت.
در این حالت در صورتی که شخصیتی خارج از دوربین هستند لازم نیستند که متحرک شوند.

کلیپ‌های انیمیشنی متفاوت از این هستند. یک بازی، یک تجربه‌ی تعاملی است بنابراین نمی‌توان از قبل چگونه حرکت کردن شخصیت‌ها و رفتار آن‌ها را پیش‌بینی کرد.
حتی تصمیمات شخصیت‌های غیربازیکن کامپیوتری نیز می‌توانند تابعی از اقدامات غیر قابل پیش‌بینی بازیکن انسانی باشد.
به این ترتیب، کلیپ‌های انیمیشنی مربوط به بازی تقریبا هیچ‌گاه از مجموعه‌ای از فریم‌های طولانی و به هم پیوسته تشکیل نمی‌شوند.
درعوض، حرکت شخصیت بازی باید به تعداد زیادی حرکات ریز تقسیم شود. 
منظور از کلیپ‌های انیمیشنی این حرکات کوتاه و یکتا است.

بنابراین هر کلیپ به صورتی طراحی شده است که یک عمل کاملا مشخص را انجام دهد. برخی از این کلیپ‌ها به گونه‌ای طراحی شده اند که بتوان آن را به صورت حلقه شونده تکرار کرد.
به عنوان مثال چرخه‌ی راه‌رفتن یا دویدن می‌توانند از این نوع کلیپ‌ها باشند.
و حرکاتی مانند پریدن یا دست تکان دادن از نوعی هستند که تنها یک‌بار پخش می‌شوند.

بنابراین به طور کلی حرکات هر شخصیت بازی معمولا به هزاران کلیپ تقسیم می‌شود. \cite{GameEngineArchitecture}

\section{ترکیب انیمیشن}

اصطلاح ترکیب انیمیشن به هر تکنیکی اطلاق می‌شود که در آن بیش از یک کلیپ انیمیشن در ژست نهایی کاراکتر سهیم می‌شود.
به صورت دقیق تر در این عمل دو یا چند ژست برای ایجاد یک ژست خروجی برای اسکلت شخصیت، با یکدیگر ترکیب می‌شوند.
همانطور که در بخش قبل گفته شد، کلیپ‌های انیمیشنی، کلیپ‌های کوتاه و یکتایی هستند. با استفاده از روش ترکیب ‌می‌توان مجموعه‌ای از کلیپ‌های انیمیشنی را با یکدیگر ترکیب کرد تا مجموعه‌ی جدیدی از انیمیشن‌ها را بدون نیاز به ایجاد دستی و از پایه‌ی آن ها تولید کنیم.
به عنوان مثال، با ترکیب یک انیمیشن راه رفتن آسیب دیده با راه رفتن بدون آسیب دیدگی، می‌توانیم سطوح مختلفی از آسیب دیدگی در هنگام راه‌رفتن را به وجود آوریم.
از ترکیب می‌توان برای درون‌یابی بین حالات مختلف چهره، حالت‌های مختلف بدن و حالت‌های مختلف حرکتی استفاده کرد.
از ترکیب می‌توان برای یافتن یا حالت میانی بین دو حالت شناخته شده در زمان‌های مختلف استفاده کرد. این‌کار زمانی استفاده می‌شود که بخواهیم ژست یک شخصیت را در نقطه‌ای از زمان پیدا کنیم که دقیقا با یکی از فریم‌های نمونه موجود در داده‌های انیمیشن مطابقت ندارد.
همچنین می‌توانیم از ترکیب موقتی انیمیشن برای انتقال هموار از ک انیمیشن به انیمیشن دیگر، با ترکیب تدریجی انیمیشن مبدا به مقصد در مدت زمان کوتاهی استفاده کنیم.