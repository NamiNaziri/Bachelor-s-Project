\section {
    \lr{OpenGL}
    }

\lr{OpenGL}
یک واسط برنامه نویسی کاربردی  
\LTRfootnote{API}
است که با فراهم کردن توابع مختلف به توسعه‌دهندگان امکان دستکاری گرافیک و تصاویر را می‌دهد.
\lr{OpenGL} 
یک کتابخانه‌ی رندرینگ است.
یک "شئ" به خودی خود در
\lr{OpenGL} 
مفهومی ندارد
و به صورت مجموعه‌ای از مثلث‌ها و حالات مختلف درنظر گرفته می‌شود. بنابراین  
وظیفه‌ی ما است که بدانیم چه شئ‌ای در کدام قسمت صفحه رندر شده است. این کتابخانه تنها وظیفه‌اش، کشیدن تصاویری که است که می‌خواهیم به تصویر کشیده‌شوند.
در این صورت اگر می‌خواهیم تصویری را به‌روزرسانی کنیم و یا به عنوان مثال شئ‌ای را تحرک دهیم باید به 
\lr{OpenGL}
درخواست دهیم که صحنه را دوباره‌برای ما رندر کند.
\cite{KhronosUsingOpenGL}

به صورت کلی 
\lr{OpenGL}
را می‌توان یک ماشین حالت بزرگ درنظر گرفت. هر حالت شامل مجموعه‌ای از متغیر‌ها است که نحوه‌ی عملکرد
\lr{OpenGL}
را مشخص می‌کند. 
به مجموعه‌ی این حالت‌ها 
\lr{OpenGL context}
نیز می‌گویند. 
در واقع  
\lr{context}
را می‌توان یک شئ درنظر گرفت که کل
\lr{OpenGL}
را دربر می‌گیرد. عموما تمامی تغییرات، روی 
\lr{context}
فعلی اعمال می‌شود و سپس رندر می‌شود.
\cite{KhronosUsingOpenGL} \cite{LearnOpenGL_GettingStarted}


\section{سایه‌زنی فونگ}

